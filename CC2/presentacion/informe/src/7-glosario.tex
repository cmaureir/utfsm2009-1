\begin{description}
	\item[Función de Bessel]
	Son soluciones canonicas $y(x)$ de la ecuación diferencial de
Bessel:
$$
x^2 \frac{d^2 y}{dx^2} + x \frac{dy}{dx} + (x^2 - \alpha^2)y = 0
$$
	\item[$J_0$]
	Función de Bessel de primera especie y orden 0. Para las soluciones de orden entero es
posible definir la función $J_0(x)$ por su expansión en serie de Taylor en
torno a $x = 0$:
$$
J_0(x)= 1-\frac{x^2}{2^2}+\frac{x^4}{2^2 4^2}-\frac{x^6}{2^2 4^2 6^2}\ldots
$$
$$
J'_0(x)= \frac{dJ_0(x)}{dx} = -J_1(x)
$$

	\item[Función continua]
	Una función $f$ es continua en el punto $a$, si $f$ está definida en algún intervalo abierto que contenga
	a $a$ y si para cualquier $\epsilon > 0$ existe un $\delta > 0$ tal que:
	$$si\ |x-a| > \delta\ entonces\ |f(x)-f(a)|< \epsilon$$
	
	\item [Malla]
	La malla de la partición:
	$$x_0 < x_1 < x_2 < \ldots < x_n$$
	es el largo del subintervalo mas largo:
	$$max|x_i - x_{i-1}|: i=1\ldots n$$
	Si la malla tiende a cero, podemos tener una suma de Riemann, lo que conlleva a una integral de Riemann.

\end{description}
