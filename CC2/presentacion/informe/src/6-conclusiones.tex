Se ha hecho aquí un intento, para crear un caso para resolver el gran problema de alimentación
de la rutina para la solución de sistemas de ecuaciones diferenciales ordinarias simultáneas.
 Esta tesis que hemos estudiado, tiene una sólida base en la experiencia
operativa del Grupo de Matemática Aplicada en los Laboratorios RCA en Princeton, Nueva Jersey.
Las rutinas que estudiamos, contienen múltiples ventajas del control
automática de errores y modificación del intervalo, y se considera
que incluso superan a las ventajas aparentes de la cuadratura de Gauss.
 Esto en sí mismo era suficiente para el procesamiento de una gran clase de
problemas.

Los ejemplos anteriores demuestran que muchos problemas pueden ser expresados en términos de la solución
de sistemas de ecuaciones diferenciales ordinarias simultáneas. Una buena rutina de propósito general
para la solución de estos sistemas proporciona una poderosa herramienta para el procesamiento de estos
problemas. Esto es cierto desde el punto de vista de la facilidad de la programación, la facilidad de
depuración y reducción al mínimo de tiempo de computadora.
