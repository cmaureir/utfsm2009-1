\subsection{Integrales definidas}
En el presente trabajo, se toma como un nuevo ejemplo, una integral general definida de la forma:

\begin{eqnarray}
	I &=& \int_{a}^{b} f(t) dt
\end{eqnarray}

El enfoque estándar para realizar esta evaluación sería seleccionar primero una \emph{fórmula de cuadratura},
pero ¿Qué son las fórmulas de cuadraturas?.

Sea $f(x)$ una función continua definida en el intervalo $[a, b]$.
Nuestro objetivo será encontrar fórmulas aproximadas para calcular la integral $\int_{a}^{b}f(x) dx$.
En caso de conocer la primitiva $F(x)$ es evidente que podemos encontrar el valor exacto de la integral
utilizando el Teorema fundamental del cálculo integral: $\int_{a}^{b}f(x) dx = F(b) - F(a)$.
Sin embargo no siempre ésto es posible.

Por ejemplo, para la funcion $f(x) = e^{-x^{2}}$
no existe ninguna primitiva que podamos escribir utilizando funciones elementales.

Algunas fórmulas de cuadraturas son:
\begin{itemize}
	\item Fórmulas de los rectángulos.
	\item Fórmulas de  los trapecios.
	\item Método de Simpson.
	\item etc.
\end{itemize}

Volviendo a nuestro tema principal, el tipo de solución es normalmente realizado sobre la base de que conocemos
la primitiva, lo cual presenta una  facilidad de programación.
Entonces, se deben definir una malla adecuada de puntos para aplicar la fórmula de cuadratura.
El intervalo de malla seleccionado es casi siempre uniforme y acotado.

La principal dificultad asociada con este enfoque es, la determinación de un intervalo apropiado de integración.
Los límites de error habituales involucran las mayores derivadas de la función del integrando, $f(t)$.
Esta función suele ser tan complicada que un análisis de, por ejemplo, la quinta derivada es totalmente impracticable o,
por decir lo menos, poco economica.
Por la misma razón, es generalmente imposible especificar un intervalo variable de integración para aprovechar las
características variables de $f(t)$ sobre el intervalo de integración.

El método común para superar estas dificultades,
es llevar a cabo la integración dos veces con el doble del número de puntos en la segunda integración.
Luego, la precisión se puede afirmar, según el acuerdo de los dos resultados.
Una desventaja de este procedimiento es que un mínimo de tres veces más evaluaciones del integrando,
son necesarias como habría sido necesario un conocimiento a priori.
La necesidad práctica de utilizar una malla uniforme normalmente puede introducir un factor adicional de tres o cuatro.
Así que las evaluaciones más un orden de magnitud de la integral que se realizan son básicamente necesarias.
Este es un factor que no puede ser ignorado incluso en los computadores de hoy en día de alta velocidad.
Cabe señalar que en el caso de las integrales múltiples de orden N, esto se convierte en N órdenes de magnitud.

Un enfoque ``preferido'' es la de formar una ecuación diferencial mediante la definición de

\begin{eqnarray}
	I(x) = \int_{a}^{x} f(t) dt
\end{eqnarray}
y diferenciando
\begin{eqnarray}
	\frac{dI}{dx} = f(x),
\end{eqnarray}
con
\begin{eqnarray}
	I(a) = 0
\end{eqnarray}
entonces
\begin{eqnarray}
	I(b) = I
\end{eqnarray}

que sería igual que $(1)$, proporcionando el resultado deseado.
Debido a la naturaleza especial de la función derivada (sólo depende de x),
sólo una evaluación de la función es necesaria por paso, en lugar de dos para las derivadas de las funciones más general.
Una buena rutina de propósito general para las ecuaciones diferenciales contendrá los medios para tomar ventaja de este hecho.

Las ventajas de este enfoque es que la rutina de seguimiento continuo monitorea la presición de la integración y
aumenta el intervalo de integración, cuando sea posible.
Una malla variable apropiada es construida en línea con sólo una evaluación del integrando por cada paso,
mientras que la precisión adecuada se mantiene.
Un orden de magnitud en el tiempo de cálculo se obtiene de las integrales individuales;
dos órdenes de magnitud para las integrales dobles.

Este enfoque se pueden programar fácilmente y depurado.
Una vez más, una buena rutina de propósito general contiene los medios para asegurar que el punto final
(x = b) estará en la malla variable construida por la rutina.
