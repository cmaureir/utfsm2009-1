\subsection{Otras aplicaciones}

Las aplicaciones anteriores están destinadas a ser meramente ilustrativos, de poder resolver
el problema de una rutina de propósito general, para la solución de ecuaciones simultáneas
diferenciales ordinarias de primer orden. Ellos fueron seleccionados debido,
en parte, a que estaban entre los más evidentes de estas aplicaciones. Algunas
aplicaciones adicionales son las que nombraremos a continuación,
a modo de ejemplo, para indicar el mayor ámbito de aplicación de este enfoque general.
Existe cierta coincidencia en la lista, pero se cree mejor ser un poco
redundante a que, posiblemente, se le reste importancia a algunos temas.

\begin{enumerate}
	\item Múltiples integrales definidas\\
		Es evidente que los métodos de la sección 3 se puede generalizar para el
		tratamiento de las integrales múltiples. Las ventajas se multiplican en este caso,
		como se indica allí.

	\item Ecuaciones diferenciales derivadas en términos de un parámetro del
problema.\\
		En muchos casos, de decir, el rito esencial, es posible derivar una ecuación diferencial
		en función de un parámetro del problema y, posteriormente, resolver esta ecuación
		diferencial numéricamente (o, en casos excepcionales analíticamente!).
		Esto normalmente elimina toda una dimensión del problema.
		El ejemplo de la sección 2 es un ejemplo de esta técnica.		

	\item Ceros en funciones no-lineales (root-tracing) \\
		Una aplicación interesante de la técnica anterior se produce en la determinación de los loci
		de ceros de funciones no lineales.
		La información detallada en relación con este tema está disponible en $ [1] $.

	\item Equipotenciales, isotérmicos, campos lineales, etc.\\
		Las ecuaciones diferenciales de las curvas de los valores de la
		función constante pueden ser determinadas a través de la raíz de
		trazado de la técnica. Similares ecuaciones diferenciales
		para las trayectorias ortogonales también se obtiene fácilmente.

	\item  Derivadas dadas implícitamente en ecuaciones diferenciales
ordinarias.\\
		Las rutinas de propósito general están diseñados para resolver ecuaciones de la forma (1) con las
		expresiones derivadas dado de forma explícita. Si esto es imposible de alcanzar en una aplicación
		dada, por lo general hay alternativas. Las derivadas pueden ser la solución de un sistema lineal
		de ecuaciones algebraicas. En general, no se debe intentar producir las soluciones explícitas,
		sino más bien, el sistema lineal debe ser resuelto numéricamente en cada paso de integración.
		Si el derivado se da implícitamente en forma no lineal $G(x,y,y') = 0$, la técnica de trazado
		de la raíz puede ser utilizado para producir una solución en línea de la ecuación no lineal.

	\item Integrales oscilatorios.\\
		Graves dificultades de computación se producen cuando el integrando de una integral es muy oscilante.
		En algunos casos, es imposible obtener resultados significativos en todos los de una integración directa.
		Una técnica eficaz para el alivio de esta dificultad es deformar el camino de la integración en el plano
		complejo. Lo que se desea es el camino de integración de arte que a lo largo de la fase de la función
		integrando es constante. La raíz de la técnica de rastreo permite la determinación simultánea de la ruta
		adecuada de la integración, junto con la integración en sí, ver $[2]$.

	\item Problemas del valor de dos puntos de la frontera.\\
		Dos problemas de límite de valor de punto se suelen resolverse a través de la solución iterativa de
		problemas de valor inicial. Algunas técnicas que giran alrededor de la solución simultánea de las ecuaciones
		diferenciales relacionadas con acelerar el proceso de convergencia, véase $[3]$. Más significativamente,
		a veces es posible reducir el problema a un problema de valor inicial en un parámetro del problema, ver $[4]$.

	\item Ecuaciones diferenciales parciales parabólicas y hiperbólicas.\\
		En la ``marcha de tipo'' problemas de ecuaciones diferenciales parciales a menudo es conveniente para
		discretizar una o más (por lo general el espacio), pero las variables de mecanización de la marcha variable
		(normalmente el tiempo) a través de una rutina de propósito general para la solución de ecuaciones diferenciales
		ordinarias simultáneas. Se obtiene un número de sistemas de ecuaciones diferenciales ordinarias igual al número
		de puntos de malla en las variables de espacio.

	\item Ecuaciones Integrales.\\
		Muchos de los problemas expresados en términos de las ecuaciones integrales tienen representaciones equivalentes
		en términos de ecuaciones diferenciales. Esta última forma es a menudo superior desde un punto de vista de la
		informática.

\end{enumerate}
