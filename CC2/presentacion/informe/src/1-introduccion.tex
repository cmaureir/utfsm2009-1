Existen una gran cantidad de problemas de procesamiento computacional que
pueden ser formulados en términos de la solución numérica de sistemas de
ecuaciones diferenciales ordinarias, para las cuales existen poderosos métodos
de resolución para encontrar la solución.\\

La forma general de las ecuaciones a tratar son:
\begin{eqnarray}
\frac{dy_i}{dx} = f_i (x, y_1, y_2 , \cdots, y_n), \nonumber \\
		\\
y_i (x_0) = y_{i0} ,\ \ \ \ \ \	(i  = 1, 2, 3, \cdots, n). \nonumber
\end{eqnarray}

La gracia de estas ecuaciones, es que casi toda biblioteca de programación
contiene subrutinas para la solucion de sistemas de este tipo. Éstas
subrutinas normalmente pueden trabajar con diferentes niveles de precisión y
intervales de integración, configurados específicamente a la medida del
problema. Además, todo el proceso es realizado con ua eficiencia muy cercana a
la óptima.

Estás características, junto con que uno sólo debe preocuparse de programar los
valores derivados (ya que las rutinas proveen los algoritmos
para combinar los valores derivados para obtener la solución), son las razones
por lo cual la transformación de problemas en sistemas de ecuaciones
diferenciales ordinarias es muy interesante.

Más adelante veremos algunos ejemplos de la aplicación de la técnica nombrada.
