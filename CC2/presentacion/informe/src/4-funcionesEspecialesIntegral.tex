\subsection{Funciones especiales en el integrando}

Siguiendo la misma idea, podemos aprovechar ésta técnica para resolver
funciones especiales dentro de una integral. La idea es que sí estas funciones
especiales satisfacen una ecuación diferencial ordinaria, pueden ser
generadas por la solución de la ecuación diferencial definida.

Por ejemplo:

$$
	I	=	\int_0^\infty J_0 (t^2) e^{-t} dt
$$
\textit{(donde $J_0$ es la función regular cilindrica de Bessel de orden 0)}\\

Según lo visto al resolver integrales definidas, podemos hacer:
$$
	\frac{dI}{dx}	=	J_0 (x^2) e^{-x},
$$
con:
$$
	I(0) = 0
$$

Y el resultado deseado de $I(\infty) = I$.\\

Ahora normalmente uno podría proceder como ántes, y utilizar funciones de
Bessel\footnote{Son soluciones canonicas $y(x)$ de la ecuación diferencial de
Bessel: $x^2 \frac{d^2 y}{dx^2} + x \frac{dy}{dx} + (x^2 - \alpha^2)y = 0$}
 y subrutinas para la exponencial. Pero si suponemos por el momento que
no hay subrutinas para las funciones de Bessel en la biblioteca del programa,
podemos utilizar una alternativa a escribir la subrutina antes de resolver el
problema.

Ahora, está determinado (creánme) que $J_0(x^2)$ satisface la siguiente
ecuacion diferencial ordinaria de segundo orden:
$$
	\frac{d^2y}{dx^2} + (2 - \frac{1}{x}) \frac{dy}{dx} + 4x^2y = 0
$$
con 
$$
	y(0) = 1, \ \ \ \ \ \ \ \ \ \ \ \  y'(0) = 0
$$

Esta ecuación diferencial puede ser resuleta simultaneamente con el problema
anterior para evaluar la integral mientras generamos la función de Bessel
requerida. Específicamente, el problema completo es resuelto siguiendo el
siguiente sistema de ecuaciones de primer grado:

$$
	\frac{dI}{dx} = y_1 e^{-x}, \ \ \ \ \ \ \ \ \ \ \ \ \ \ \ I(0) = 0,
$$
$$
	\frac{dy_1}{dx} = y_2 ,\ \ \ \ \ \ \ \ \ \ \ \ \ \ \ \ \ \ y_1(0) = 1,
$$
$$
	\frac{dy_2}{dx} = \{^{-(2-\frac{1}{x})y_2-4x^2y_1 \ \ \ \ \ \ (x \neq
0)}_{0 \ \ \ \ \ \ \ \ \ \ \ \ \ \ \ \ \ \ \ \ \ \ (x = 0)}, \ \ \ y_2(0) = 0,
$$

Y se puede determinar facilmente un estimador del resto para proporcionar un
criterio para el término de la integración.

Esta última formulación del problema tiene varias ventajas:
\begin{enumerate}
	\item Evitamos la necesidad de escribri una subrutina especial para la
función de Bessel de esta aplicación en particular.
	\item La exactitud de los valores producidos por la función especial son
automáticamente monitoriades y son consistentes con la exactitud requerida
para la integral.
	\item Normalmente habrá menos tiempo de computación por cada evaluación de
la función en un factor de 4 o 5 más que la misma función evaluada por una
subrutina.
	\item Por último, habrá una ganacia en general por el acercamiento
realizado de un factor cerca de 25.
\end{enumerate}

Con la disponibilidad de una rutina de propósito general para la resolucion de
ecuaciones diferenciales ordinarias, la solución del sistema es fácilmente
programable y corregible.



