Luego de programar los algoritmos en matlab se procedió,
a ejecutarlos para ver los resultados, y notar si se existía un
valor parecido en ambas situaciones.

\begin{tabular}{llcll}
	\textbf{Newton:} & $y_{n+1} = {-0.8159 \brack -0.3694}$ & &\textbf{Lipschitz:} & $y_{n+1} = {-0.8133 \brack -0.3703}$ \\
\end{tabular}

Con respecto al tiempo que se empleo en realizar el cálculo
de cada algoritmo, nos dimos cuenta que no siempre era el mismo,
esto tiene completa relacion porque no realizamos los cálculos
en una máquina \emph{Real Time}, por lo tanto lo que decidimos
fue poder realizar 10 ejecuciones y promediar un valor,
para poder obtener un valor cercano al real.

A continuación se muestra una tabla con los tiempos (en segundos) 
que resultaron de las 10 ejecuciones de nuestros algoritmos.
\begin{center}
\begin{tabular}{|c|c|c|c|c|c|c|c|c|c|c|c|}
	\hline
	Método & 1 & 2 & 3 & 4 & 5 & 6 & 7 & 8 & 9 & 10  \\\hline 
	Lipschitz (s) & 25.15 & 21.46 & 21.03 & 23.00 & 23.63 & 23.59 & 22.69 & 21.83 & 25.18 & 24.04  \\\hline
	Newton (s)    & 23.85 & 23.66 & 23.23 & 21.50 & 21.45 & 22.47 & 23.25 & 22.36 & 24.33 & 21.83  \\\hline
\end{tabular}
\end{center}
Finalmente tenemos el promedio:
\begin{tabular}{llcll}
	\textbf{Newton:} & 22.793 segundos & &\textbf{Lipschitz:} & 23.160 segundos \\
\end{tabular}
