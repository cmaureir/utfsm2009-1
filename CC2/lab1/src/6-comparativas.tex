
\scriptsize
\begin{tabular}[b]{|l|l|l|}
\hline
a	&	\textbf{Newton}	&	\textbf{Lipschitz}	\\
\hline
0	&	for n=1 a N	&	lo mismo \\
1	&	hace $t_{n+1} = t_n +h$	& lo mismo	\\
2	& $\Phi_k = \Phi_k (u,v)$	& lo mismo	\\
3	& $z_1(u, v) = u - \Phi_1 (u,v)$	& lo mismo	\\
4	& $z_2(u, v) = u - \Phi_2 (u,v)$	& lo mismo	\\
5	& $v = 0; (u_0, v_0) = (y^1_n, y^2_n);$	& lo mismo	\\
6	& while $|z_1(u_v, v_v)| > \epsilon$ or $|z_2(u_v,v_v) |> \epsilon)$ & lo mismo	\\
7	& $ \left[ {\begin{array}{c} u_{v+1}  \\ v_{v+1}\\ \end{array} }\right]  = 
\left[ {\begin{array}{c} u_{v}  \\ v_{v}\\ \end{array} }\right]  +
\left[ {\begin{array}{cc} 1-\frac{\delta \Phi_{1}}{\delta u} (u_{v}, v_{v}) & -\frac{\delta \Phi_{1}}{\delta v} (u_{v}, v_{v})   \\
 -\frac{\delta \Phi_{1}}{\delta u} (u_{v}, v_{v})  & 1-\frac{\delta \Phi_{2}}{\delta
v} (u_{v}, v_{v}) \\ \end{array} }\right]^{-1}
 \left[ {\begin{array}{c} -z_{1} (u_{v}, v_{v})  \\ -z_2 (u_{v}, v_{v})\\ \end{array} }\right] 
$ & $u_{v+1} = \Phi_1 (u_{v}, v_{v}); v_{v+1} = \Phi_2 (u_v, v_v);$	\\
8	& $v = v+1$ & lo mismo	\\
9	& end while & lo mismo	\\
10	& $(y^1_n , y^1_n+1) = (u_v, v_v)$ & lo mismo	\\
11	& end for & lo mismo	\\
\hline
\end{tabular}
\\
\normalsize


Claramente podemos observar que las diferencias radican en el tipo de
resolución, en el cual el algoritmo con iteraciones de Lipschitz es más simple
que realizar las iteraciones de Newton.

La cantidad de operaciones que realiza una iteración de Lipschitz quedan
definidas por resolver $\Phi_1 (x,y)$ y $\Phi_2 (x,y)$. En cambio, una
iteración de Newton en este caso debe calcular el plano tangente respectivo a
cada función $y$ ($y_1$ e $y_2$), y luego resolver el intercepto con el plano
del eje $X$ para obtener las soluciones de la ecuación que utiliza para ir
acercándose al valor de $y_{n+1}$.

Ahora, el método de Lipschitz, a pesar de ser más simple, requiere de muchas
más iteraciones que el método de Newton, lo cual aumenta el tiempo de
resolución del problema en \textbf{inserte valor acá}, a lo cual si uno le suma la
poca capacidad de paralelización en la resolución del problema, puede llegar a
ser más costosa computacionalmente. 
\\
Otra diferencia fundamental de estos dos métodos, es que para poder utilizar
el método iterativo de Lipschitz, primero hay que verificar que la función
converja de la siguiente forma:\\
Dado $y'=f(x,y)$, despejamos $f(x,y)$ para lograr que en un costado de la
ecuación quede solamente $x$ y la igualamos a $x=g(x,y)$
\begin{itemize}
	\item Si $|g'(x,y)|< 1$ Converge linealmente
	\item Si $|g'(x,y)|= 0$ Converge cuadraticamente.
	\item Si $|g'(x,y)|> 1$ Diverge linealmente
\end{itemize}

Si la condición de convergencia no se cumple, simplemente no podríamos
utilizar este método para resolver el problema, ya que nos quedaríamos iterando
infitamente, alejándonos cada vez más de la solución.
