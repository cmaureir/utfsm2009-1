\documentclass{beamer}
\usepackage[dvips]{epsfig}
\usepackage[dvips]{graphicx}
\usepackage[utf8]{inputenc}
\usepackage{verbatim}
\usepackage{graphicx}
\usepackage{moreverb}
\let\verbatiminput=\verbatimtabinput
%\documentclass[xcolor=DarkRed]{beamer}
%\usepackage{beamerthemesplit}
\usetheme{CambridgeUS}
%\documentclass[xcolor=DarkRed]{beamer}

\title{``Elian y Cia., Ltda.''} 
\subtitle{Sistemas y Organizaciones}
\author{Javier Olivares, Cristi\'an Maureira, Rodrigo Fern\'andez}
%\author{jolivaro,rfernand,cmaureir}
\titlegraphic{\includegraphics[width=3.5cm]{images/empresa}}


\date{\today}

\begin{document}
\frame{\titlepage}
\frame{\tableofcontents}

\section{La Empresa}
\frame{
\frametitle{Resumen}
% Resumen del informe, para contar que onda la empresa
\begin{itemize}
	\small
	\item \emph{Rubro}, Producción, venta y reparaciones de joyas
	\item \emph{Experiencia}, Lleva más de veinte años en la Quinta Región.
	\item \emph{La empresa}, Dispone de talleres propios siguiendo procesos a cargo de orfebres especializados.
	\item \emph{Sucursales}, Posee 4 sucursales de ventas y una casa central donde tienen todo el proceso de producción.
	\item \emph{Personal}, Aproximadamente 30 personas.
	\item \emph{Principal Característica}, Preocupada por el bienestar de sus trabajadores.
	\item \emph{Valor fundamental}, Confianza.
	\item \emph{Áreas de trabajo},
	\begin{itemize}
		\item Ventas (vendedores de cada sucursal)
		\item Producción (la mayoría orfebres)
		\item Administración (secretarias, informática y gerencia).
	\end{itemize}
\end{itemize}
}

\section{Teorías}
\frame{
\frametitle{Teorías utilizadas}
\begin{itemize}
        \item \textbf{Luther\ Gullick:} \emph{``PODSCORB''}\\
        \item \textbf{Adam\ Smith:} \emph{``La Riqueza de las Naciones''}\\
        \item \textbf{\'Emile\ Durkheim:} \emph{``Hechos sociales''}\\
        \item \textbf{Frederick\ Taylor:} \emph{``Administración Científica''}
        \item \textbf{Mary\ Parker\ Follet:} \emph{``El Nuevo Estado''}\\
        \item \textbf{Chester\ Barnard:} \emph{``Influencia de factores sicológicos y sociales en la efectividad de la organización''}\\
        \item \textbf{Fred\ Emery:} \emph{``Sistemas Sociotécnicos''}\\
\end{itemize}
}

\section{Análisis}
\frame{
\frametitle{Luther Gullick}
\emph{``PODSCORB''}\\
\begin{itemize}
	\item Autoridad central.
	\item Organización en crecimiento.
	\item En busca del bienestar de los trabajadores.
	\item Poca rotación de personal.
	\item Sin etapas de presupuesto.
\end{itemize}

%Desarrollo breve
}

\frame{
\frametitle{Adam Smith}
\emph{``La Riqueza de las Naciones''}\\
\begin{itemize}
	\item Especialización en el trabajo.
	\item Trabajadores con grandes libertades.
\end{itemize}
%Desarrollo breve
}

\frame{
\frametitle{\'Emile Durkheim}
\emph{``Hechos sociales''}\\
\begin{itemize}
	\item Gran cercanía entre los trabajadores.
	\item Ambiente ``familiar'' en la organización.
\end{itemize}
%Desarrollo breve
}

\frame{
\frametitle{Frederick Taylor}
\emph{``Administración Científica''}\\
\begin{itemize}
	\item Labores específicas para cada trabajador.
	\item Pero además, tienen libertades para poder realizar otras tareas.
\end{itemize}
%Desarrollo breve
}

\frame{
\frametitle{Mary Parker Follet}
\emph{``El Nuevo Estado''}\\
\begin{itemize}
	\item Ambiente de cooperación y solidaridad favorable para el correcto funcionamiento de la
	organización.
\end{itemize}
%Desarrollo breve
}

\frame{
\frametitle{Chester Barnard}
\emph{``Influencia de factores sicológicos y sociales en la efectividad de la organización''}\\
\begin{itemize}
	\item Coordinación en pos del logro de los objetivos de la empresa.
	\item Presencia de un cierto fin común.
\end{itemize}
%Desarrollo breve
}

\frame{
\frametitle{Fred Emery}
\emph{``Sistemas Sociotécnicos''}\\
\begin{itemize}
	\item Interés hacia la tecnología.
	\item En un proceso de transición.
	\item Roles flexibles y redundancia de funciones.
	\item Minimización de la burocracia.
\end{itemize}
%Desarrollo breve
}

\section{Conclusiones}
\frame{
%\frametitle{Conclusiones Generales}
\begin{center}
	\huge
	\emph{Conclusiones Generales}
\end{center}
}


\end{document}
