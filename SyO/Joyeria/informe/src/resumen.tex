%Explicar el contenido del trabajo en unas pocas líneas.

%Hablar de la empresa, dar como una introduccion
%seria ideal poner la informacion que tenemos de cuando hablamos con mauricio,
%la primera vez que fuimos

Este trabajo consiste en un análisis organizacional desde el punto de vista de 7 teorías, muy importantes
para el estudio de las organizaciones, de la micro-empresa ``Elian Y Cía., Ltda.''.

Esta micro-empresa se encarga de la producción, venta y reparaciones de joyas, lleva más de veinte años en
la Quinta Región. Dispone de talleres propios siguiendo procesos a cargo de orfebres especializados.
Elaboran todo tipo de cadenas, anillos, argollas de matrimonio y colgantes (de oro y plata).
Se han preocupado por implementar nuevas tecnologías, ofreciendo fotograbado a color en Plata, Oro y
Acero.

Posee 4 sucursales de ventas y una casa central donde tienen todo el proceso de producción. El personal
no son más de 30 personas, es una organización exigente, pero preocupada por el bienestar de sus
trabajadores. Como se manejan objetos pequeños de gran valor, la confianza en los trabajadores y buenos
métodos de control son primordiales para la empresa. Por ello, el ambiente entre los colaboradores es muy
familiar, donde la alta gerencia está conformada por el jefe de todos, Sr. Iván González y el gerente de
ventas y producción, su hijo Iván González.

Las áreas de trabajo en la empresa se pueden identificar como
ventas (vendedores de cada sucursal), producción (la mayoría orfebres) y administración (secretarias, 
informática y gerencia).
