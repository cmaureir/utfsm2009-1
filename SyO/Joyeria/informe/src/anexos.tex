%Anexar encuestas realizadas (transcritas en sus partes esenciales, no textualmente).
%Debe haber al menos 10 encuestas.
%Se debe identificar al encuestado y adjuntar material multimedia que pruebe que la encuesta realmente
% se realizó (entregar en un CD o DVD).
%Se puede anexar también cualquier material que los alumnos estimen pertinente.i

\subsection{Preguntas de la Entrevista}

\begin{enumerate}
	\item ¿Qué cargo tiene actualmente?
	\item ¿Cuál sería su principal tarea asignada?
	\item ¿A cuántas personas conoce de la misma empresa?\\
	      ¿Cómo encuentra su relación con sus subordinados, compañeros y Jefes en el trabajo?
	\item ¿Se siente cómodo en el ambiente de trabajo?\\
	      ¿Si es así, que es lo que más le agrada del mismo?, en caso contrario\\
	      ¿Qué es lo que más le desagrada del mismo?
	\item ¿Cuánto tiempo lleva trabajando en la compañia?
	\item ¿Quién es el encargado de tomar las decisiones dentro de la empresa?,\\ %gullick
	      ¿Encuentra que su opinión es considerada dentro de ellas?. %gullick
	\item ¿Existe dentro del año alguna etapa de presupuesto?, ¿en qué consiste?. %gullick
	\item Con respecto a las relaciones que existen en ésta organización,\\
	      ¿Piensa usted que ésta organización se preocupa más del aspecto economico o social? % (Schmidt: riqueza naciones)
	\item De una escala de 1 a 10, ¿Cómo calificaría su trabajo? ¿Por qué? % "Taylor:administracion cientifica"
	\item ¿Siente que todos las personas con las que trabaja se les trata por igual? %(Follet: nuevo estado)
	\item ¿Coopera con sus compañeros de trabajo? ¿De que forma? %(Barnard: "influencia de factores sicologicos y sociales en la efectividad de la organizacion")
	\item ¿Cómo es su relación con las herramientas tecnológicas que se utilizan en la organización? %(Fred Emery: Sistemas sociotecnicos)
	\item En pocas palabras, ¿Cómo definiría usted la organización en la cual trabaja?

\end{enumerate}

\newpage

\subsection{Respuestas}

%1- Mauricio
\textbf{Entrevista 1}\\
\begin{enumerate}
	\item Sistemas Computacionales.
	\item Control de Inventario y Ventas.
	\item Conoce a todos. En general tiene una buena relación con los demás.
	\item Se siente cómodo en el ambiente de trabajo, pero encuentra que falta más personal y comunicación con gente de su misma área (esta solo).
	\item Medio año a tiempo completo y de contratado a tiempo parcial desde hace 6 años.
	\item Su jefe, Iván Gonzales. Su opinión si es tomada en cuenta en su área de trabajo.
	\item No existe un presupuesto anual. Las cosas se compran a medida que se van necesitando.
	\item Económico.
	\item Nota 7: Porque no ha sido todo bueno. Varios inconvenientes por falta de preocupación de los jefes hacia el área en que trabaja.  No se da espacio a otra cosa que no sea de ventas.
	\item Difícil responder (hijo del jefe). Siente que si se les trata por igual a los demás.
	\item Muy poca cooperación y trabajo en equipo como tal (trabaja solo).
	\item 200\% relacionado con las tecnologías.
	\item Organización seria, trabajo duro, antigua y poco flexible, toma de decisiones centralizada (y que le cuesta mucho ceder el mando), con expectativas de crecimiento. Le falta desarrollo en las áreas tecnológicas, hay muchos procesos que todavía son llevados en papel.
\end{enumerate}
\newpage
%2- Paola Miranda
\textbf{Entrevista 2}\\
\begin{enumerate}
	\item Orfebre, Banco de Trabajo, casting, vaciado de piezas en anillos...
	\item Sacar compostura, pulir, de todo un poco.
	\item Conoce y se lleva bien con todo el mundo (excepto uno).
	\item Se siente cómoda en el trabajo. Estudió Laboratorista Dental, y es orfebre de familia.
	\item 10 años en Septiembre.
	\item Todas las decisiones a través de Iván Gonzales, con las cuales a veces esta de acuerdo, y los conflictos los resuelve personalmente con el.
	\item No existe un presupuesto fijo. Las cosas se compran a medida que se van necesitando semanalmente. A veces los consultan sobre que necesitan.
	\item Económico. A pesar de ello, en lo social esta satisfecha, siente que existe una buena preocupación por ella (única mujer en el área de producción). Por los demás no está muy segura.
	\item Nota 8: Le gusta su trabajo, y trata de hacer su trabajo pensando en el cliente.
	\item No siente que a todos se les trate por igual. A algunos se les trata mas ``brusco'' o se les exige más de lo que pueden realizar en el día. A llegado a pensar que los sobre explotan.
	\item Bien cooperadora con los demás, siempre ayuda y comparte sus cosas con los demás.
	\item No se relaciona con las tecnologías. No le da el tiempo en el día para dedicarse a ello. Todo se hace por papel.
	\item Buena, con harto crecimiento, avanzando hacia los trabajadores.
\end{enumerate}

\newpage
%3- Gladys
\textbf{Entrevista 3}\\
\begin{enumerate}
	\item Jefe de personal.
	\item Inventario, preocuparse que el local esté siempre limpio, orden y el buen uso de uniformes y que se realicen las ventas.
	\item Conoce a todos (+ de 20 personas). Buena relación con jefes y compañeros.
	\item Cómoda con su ambiente de trabajo, le agrada porque comparte con sus compañeros cosas no sólo del trabajo.
	\item 10 años.
	\item Iván Gonzales, después Iván Gonzales Jr., y después ella. Se siente considerada y respetada.
	\item No hay un presupuesto destinado para el año. Lo que se va necesitando se va comprando.
	\item Económico. En sus comienzos, la empresa era más social.
	\item Nota que le pondría a su trabajo, 9: Porque es responsable y dedicada a su trabajo. Nota que le pondría a su cargo, 6: Porque a veces hace mucho y se le da poco (monetariamente hablando).
	\item No se les trata a todos por igual. Entre compañeros si, pero el trato es diferente dependiendo del área en que uno se encuentre (ventas, recepción, producción, etc.).
	\item No tiene problemas para compartir y trabajar en equipo.
	\item Sabe que se está implementando un sistema con nuevas tecnologías y se siente capacitada para usarlos (han habido cursos de capacitación).
	\item Organización buena y una buena administración de parte del jefe.
\end{enumerate}

\newpage
%4- Germán
\textbf{Entrevista 4}\\
\begin{enumerate}
	\item Maestro Joyero.
	\item Macetar y argollas de compromiso.
	\item Buena relación, conoce que hay más de 20 personas.
	\item Le agrada, exceptuando cuando le llega el trabajo tarde y debe quedarse hasta tarde.
	\item 12 años.
	\item El jefe, Don Iván.
	\item Se le compra lo que necesita.
	\item Económico y social van de la mano. Trabajo directo y muy unido.
	\item Nota 10: experto en el trabajo, le gusta que el trabajo, y que sea individual, con altas libertades.
	\item Siente que a todos se les trata por igual.
	\item No tiene problemas con ayudar a los demás en el trabajo.
	\item Sólo con la máquina macetadora de su trabajo.
	\item Horizontal en relaciones personales.
\end{enumerate}


\newpage
%5- Rosa 
\textbf{Entrevista 5}\\
\begin{enumerate}
	\item Vendedora
	\item Mantener el local, que no falte la mercadería, etc.
	\item Conoce a todas, buenas relaciones.
	\item Cómoda, buena relación con compañeros.
	\item Hace 2 meses se reintegró a la empresa, si no se hubiera salido anteriormente llevaría 10 años en ella.
	\item Distinto en los locales y el taller (producción). %Se decios compra metales.
	\item Se piden cosas que se necesitan directamente al jefe.
	\item Económica.
	\item Nota 5: mucha responsabilidad.
	\item Trato igualitario para todos.
	\item Cooperación con otros cuando hay mucho trabajo.
	\item Recién se integra el uso de Computadoras. Esperan futuras capacitaciones.
	\item Buena, estable y a la cabeza es un jefe.
\end{enumerate}


\newpage
%6- María Ángela
\textbf{Entrevista 6}\\
\begin{enumerate}
	\item Vendedora.
	\item Atención del público.
	\item Es nueva, y tienes buenas relaciones.
	\item Conversar con gente.
	\item 2 meses.
	\item El jefe y no hay opinión.
	\item No conoce presupuesto.
	\item Económico, ventas.
	\item Nota 6 en desempeño y nota 5 para la importancia del cargo.
	\item Depende de la confianza con el jefe.
	\item Ningún problema con la tecnología.
	\item Estable.
	\item (No emite opinión).
\end{enumerate}


\newpage
%7- Joahna
\textbf{Entrevista 7}\\
\begin{enumerate}
	\item Vendedora.
	\item Vender.
	\item Conoce a todos, buenas relaciones.
	\item Le gusta su trabajo, pero le gusta más el trabajo de orfebrería.
	\item 2 años.
	\item Las decisiones las toma el Jefe (Iván Gonzales). Si se toman en cuenta sus opiniones.
	\item No hay.
	\item Ambas.
	\item Nota 8.
	\item Hay un trato diferenciado.
	\item Si ayuda en taller (en tiempo libre).
	\item No se relaciona con Computadores.
	\item Buena, organizada, cada vez mejor, facilitando el trabajo.
\end{enumerate}

\newpage
%8- Roxana
\textbf{Entrevista 8}\\
\begin{enumerate}
	\item Vendedora.
	\item Vender.
	\item Conoce a todos, buena relación con compañeros.
	\item Cómoda con el ambiente, sólo que el horario no le gusta.
	\item 2 meses.
	\item Decisiones las toma el Jefe (Iván Gonzales). Y si se toman en cuenta sus opiniones.
	\item No lo hacen ellas.
	\item Se procura más de económico que lo social.
	\item Nota 9: Buen cargo, con gran responsabilidad.
	\item A veces, depende de la situación. Alta tolerancia.
	\item Se comparten quehaceres, trabajo en equipo.
	\item No mucha relación con computadores. Se siente capacitada.
	\item Buena empresa, solvente, sólida, bastantes estrategias. Lo mejor es la cabeza.
\end{enumerate}

\newpage
%9- Ingrid
\textbf{Entrevista 9}\\
\begin{enumerate}
	\item Secretaria.
	\item Atención del público, ver quien entra y sale de la oficina, etc.
	\item Bien pero no excelente. Conoce a todos los de la empresa personalmente o por video o voz.
	\item Le gusta su trabajo.
	\item 3 meses.
	\item Don Iván. Si, a Don Iván le gusta escuchar la opinión de los demás.
	\item Presupuesto como tal no hay. Se hace en el momento.
	\item Económica.
	\item Nota de su puesto de trabajo: 10 (en importancia). Tiene que venir hasta los sábados. En desempeño, se pone una nota 8, ya que le falta entender al 100\% la parte de ventas.
	\item Si, encuentra que los tratan por igual.
	\item Si, no tiene problemas y coopera con facilidad.
	\item No tiene problemas para usar un computador (y sólo eso).
	\item Puntual, bien organizada.
\end{enumerate}

\newpage
%10- Maribel
\textbf{Entrevista 10}\\
\begin{enumerate}
	\item Vendedora.
	\item De todo, ventas, secretariado, administración.
	\item De acá todos, y en general bien.
	\item No, le desagrada el horario.
	\item 5~6 años.
	\item Don Iván, si nos toma en cuenta nuestras opiniones.
	\item Al momento que se necesita algo se compra.
	\item Económica.
	\item Nota en desempeño personal: 8~9. Nota para su puesto de trabajo: 5.
	\item No, por sector. Taller tiene muchos privilegios.
	\item Si coopera, ayudándoles a hacer trabajos.
	\item Si, no tiene problemas con la tecnología. Les hicieron una capacitación para ello.
	\item Buena, responsable, muy vista la parte económica, muy arriba el jefe comparado con los compañeros de trabajo. Tienen eventos de esparcimiento a lo largo del año.
\end{enumerate}
\newpage
%Ivan Gonzales BIG BOSS
\textbf{Entrevista 11}\\
\begin{enumerate}
	\item Dueño de la Empresa.
	\item Visualizar las necesidades de los clientes, ver que se requiere, comprar y administrar la
		empresa.
	\item De la empresa a todos. Considera que tienen una relación bien estrecha y ``atípica'' con
		todos los de la empresa. 
	\item Si, le agrada todo lo que hace en la empresa. Lo que menos hace es la venta directa con el
		cliente.
	\item Fundador de la compañía. En 1971 comenzó a trabajar en el área de la joyería haciendo venta
		puerta a puerta y con sus colegas (Profesor Humanista). En 1987 se instaló con un negocio
		físico, para luego en 1994 integrar la fabricación de joyas, con lo cual actualmente
		cubre los procesos de producción y venta de joyas.
	\item El mismo toma las decisiones dentro de la empresa.
	\item No existe un presupuesto.
	\item Ambas, social y económica. Siempre están preocupados por la parte social. Los sueldos no
		son parejos, pero también cada uno tiene diferentes libertades.
	\item Nota 8, porque ya esta un poco cansado. Encuentra que le falta más personal. Tiene un
		horario de trabajo de 8:00 hrs a 21:30 hrs todos los días, excepto Sábado y Domingo (el año pasado
		también trabajaba los Sábados)
	\item Piensa que no mayor diferencia. Trata de ser parejos con todos, recompensando a los que
		trabajan bien.
	\item ``Hay que dar el ejemplo''. Siempre visita el trabajo que realizan todos para revisar de
		que todo esté y ellos estén bien.
	\item Si, dice manejarse bien.
	\item Organización ``viva'', tratando de lograr la satisfacción del cliente. Cada uno valora el
		trabajo que hace. Enfocados al cliente. Se respetan los sueldos en la fecha que corresponden (con
		puntualidad), que escucha las opiniones y peticiones de los demás. No hay un ambiente competitivo dentro
		de la empresa.
\end{enumerate}
\textbf{Apuntes Extra:}
\begin{itemize}
	\item Contratan una empresa contadora externa.
	\item Tienen contratos de trabajo ``universal'', de 45 horas.\\ La mayoría cumple más que eso,y
		no se pagan directamente como horas extra (se toma como forma de medición de la entrega
		de parte de los trabajadores hacia la empresa). 
	\item Solamente hay 3 personas con llaves que abren la ``casa central'' por la mañana.
	\item Hacen en el año actividades informales durante el 18 de septiembre (asado) y durante los
		primeros días de Enero (celebrando el fin de año).
\end{itemize}
\newpage
%Ivan Jr.
\textbf{Entrevista 12}\\
\begin{enumerate}
	\item ``Un poco de todo''. Product Manager. Alta gerencia.
	\item  Gestión y Venta de productos. Nuevos negocios. TIC's, Marketing.
	\item Todas las personas que colaboran con nosotros (30 aproximadamente). Buena relación.
	\item Conforme y a gusto. No hay tiempo para aburrirse.
	\item 20 años trabajando en la compañía.
	\item Director gerente (Iván Gonzales Padre). Si es tomado en cuenta.
	\item En Febrero hacen una reunión para ver los puntos del año, dejando las cosas abiertas a
		modificaciones a lo largo del año.
	\item Económico. Como responsabilidad social, orientado a resolver los problemas
		personales/familiares de los colaboradores de la empresa.
	\item Nota 8: Hace falta más recursos económicos para más ideas y proyectos.
	\item Si, se les trata a todos por igual.
	\item Trabaja a la par con ellos. Apoya a quienes lo necesiten en el área de producción.
	\item Sí, muy bien. Las cosas actualmente se han hecho con bajo presupuesto, apoyados en
		practicantes para ello (para la pagina web, etc.).
	\item Responsable, innovadora, intensa (lo que se promete se cumple), todos orientados en pos del
		bien común de la empresa.
\end{enumerate}
