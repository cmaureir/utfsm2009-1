%Cada alumno se identifica y da una conclusi?n personal acerca de los temas tratados.

\begin{itemize}
	\item Javier Olivares

		Sin duda la organización se caracteriza por su complejidad y variedad, a pesar de esto no ha sido 
		complicado encontrar ciertos parámetros y situaciones que se acercan a las propuestas por los 
		distintos pensadores revisados en el ramo y particularmente a los analizados en este informe. 
		Este trabajo me ha sido de gran ayuda para entender como se estructura una organización, y además como 
		deben ser las interacciones entre las personas y herramientas que la componen. 
		
		Otra importante conclusión es que la empresa trabaja cumpliendo con varios postulados de los autores, pero
		sin tener quizás clara conciencia de esto, lo cual valida mayormente todas las teorías y postulados 
		propuestos por los pensadores mostrándonos que a pesar del paso del tiempo todas estas siguen siendo 
		válidas y estrechamente relacionadas con la realidad del día a día de la organización.
		
		Gracias al apoyo entregado por la jefatura de la empresa, hemos podido entrevistar a trabajadores de
		distintas áreas obteniendo respuestas sinceras, logrando una mirada global uniendo las ideas de cada uno de
		los trabajadores y jefes.

	\item Rodrigo Fernández

		Es muy importante mantener un ambiente de unidad dentro de los diferentes grupos de cada organización,
		preocupándose del bienestar de los trabajadores para que se sientan acogidos y atendidos mientras
		realizan su trabajo. De esta forma, se motiva a los trabajadores por mantener ese ambiente y acogen más
		facilmente las metas y objetivos comunes de la empresa.

		Además, el otorgarles no sólo el bienestar económico, sino las libertades necesarias para que puedan
		desarrollar su personalidad y fraternizar con sus co-trabajadores, ayuda a generar mejores equipo de
		trabajo y a generar una organización orgánica.

	\item Cristián Maureira
	
		Cuando uno habla de organización, no nos damos cuenta de la importancia de tal.
		Una organización es mucho más que un Jefe con Empleados, que desarrollan una cierta actividad,
		una organización es un ser, formado por cada trabajador de ella, sin importar su cargo,
		todos son partes fundamentables, es por eso que cuando uno se refiere a la idea de organización
		estamos hablando de un conjunto de personas que tienen un mismo objetivo en común.

		Es necesario sentirse identificado con el lugar donde uno trabaja,
		por el simple hecho que el trabajo se hace mucho más agradable, y nos volvemos más
		productivos; lo que beneficia a la organización y a mi mismo.

		El trabajo realizado, nos sirvió para comprender que hacer una \emph{teoría}
		acerca de mecanismos de trabajos, o de como debe ser una organización,
		es una tarea compleja, que requiere un estudio minuciosos, a la misma organización
		ya que es necesario poder saber la impresión de cada trabajador,
		saber si las normas son correctas o no,
		darnos cuenta si la productividad aumenta o disminuye con ciertos cambios en el ambiente laboral, etc.

		Un punto importante que me gustaría recalcar,
		es el hecho de que se trataba de una organización no muy grande,
		lo que creaba un ambiente grato, pues todos se conocian mutuamente,
		y más que compañeros de trabajo, habían muchos lazos de amistad;
		lo cual es completamente positivo, por el hecho de que es mas agradable nuestro día a día.
		Al ser pequeña, también facilitaba la comunicación con compañeros de trabajo y con el mismo jefe,
		lo que nos dió a entender, que todos los trabajadores son escuchados,
		y que cada persona relacionada con la organización tiene su grado de importancia.


\end{itemize}

