%Aplicación de las teorías escogidas a la organización en estudio.
%Este análisis constituye la parte esencial del informe.
%Deben analizarse las ENCUESTAS realizadas, las observaciones hechas y también detallar
% explícitamente características de las teorías que se aplican a la
% organización, como también aquellas características que no se aplican.

\subsection{Teoría de Gullick}
	Según la teoría de Gullick en toda organización debe existir una etapa de planificación, la cual
	en esta organización esta guiada integramente por la jefatura, los entrevistados indicar que los jefes
	guían toda la labor que se debe realizar y de que manera llevarla a cabo. En cuanto a la etapa de organización se 
	establece de manera clara la autoridad del jefe, es decir, el Sr. Iván González. Se observa que la dirección 
	está guiada por la jefatura tomando decisiones para luego generar instrucciones relacionadas con las labores
	de los entrevistados. En la organización el staffing, según Gullick es una etapa necesaria e importante, se ve
	claramente que la organización ha ido creciendo y agregando nuevas condiciones al ambiente laboral
	buscando crear un ambiente ameno, cómodo y favorable para la motivación de los trabajadores de
	la empresa, tales como cocinas, televisores en los talleres, baños separados por sexo, vestidores, etc. 
	La mayoría de los empleados lleva 10 años trabajando en la organización, lo cual nos indica que
	la empresa tiene políticas de muy poca rotación de personal. Respecto a la coordinación con las
	distintas áreas de trabajo, vemos que los de cada proceso productivo no tienen mayor relación con
	las otras áreas, por ejemplo, la persona de producción de anillos menciona que no se relaciona con las
	vendedoras. Finalmente vemos que la organización no se adapta al concepto de 
	presupuesto que indica Gullick, ya que no existen etapas de presupuesto fijas, más bien se va destinando
	recursos a medida que se necesitan para herramientas y materias primas.\\

\subsection{Teoría de Smith}
	Smith nos habla de la especialización de trabajo buscando la reducción de los costos de producción,
	según el análisis que llevamos a cabo, las labores de cada uno están bien especificadas, en las
	cuales se busca que sean especialistas en su trabajo, pero es posible notar que se les da la
	libertad de aprender nuevas labores dentro del área en que se desempeña, compartiendo con sus
	compañeros que saben y hacen otras tareas. Smith además nos habla de que tanto obreros como 
	empresarios buscan el interés propio, en este caso vemos que si se da esta situación, pero es compensado
	con preocupación de parte del jefe por la situación personal de los entrevistados. \\


\subsection{Teoría de Taylor}
	El pensamiento Tayloriano no se manifiesta de manera tan general dentro de la organización, ya que vemos que no
	en todos los casos se da la especialización inmutable dentro de una cadena de producción. Cada trabajador tiene una labor
	asignada, algunas ocasiones es una labor que viene a ser complementada con la labor de otro trabajador, es aquí donde
	podemos encontrar el concepto de cadena de trabajo, pero las libertades que tienen para poder
	ayudar a los demás y cambiar las tareas que realizan de un tiempo a otro, rompe el esquema mecanicista
	planteado por Taylor. \\

\subsection{Teoría de Follet}
	Considerando las ideas de Follet vemos que los trabajadores se siente motivados dentro del
	trabajo que realizan (con algunas excepciones en el área de ventas), no existen mayores conflictos entre los
	 compañeros de trabajo, son cooperadores y preocupados por los demás, ayudando
	en lo laboral y personal. Sin duda se presentan características que Follet postuló que eran favorables para 
	el correcto funcionamiento de la organización, como la fuerza de grupo y la existencia de un fin
	común. Se da un sentimiento de unidad dentro de la empresa.\\
	
	Es posible ver un ambiente grato de trabajo, aún cuando en ocasiones existen problemas
	debido a diferencias de opinión entre los trabajadores y la jefatura. La empresa se
	caracteriza por ser puntual y por hacer trabajos de calidad buscando siempre la
	satisfacción del cliente. Es por esta razón que algunos trabajadores presentan cierto
	disgusto al recibir trabajos a última hora y tener que quedarse más tiempo del horario
	laboral para completar rapidamente el trabajo comprometido al cliente.\\

	La empresa presenta una característica que Follet indica como necesaria que es la
	cooperación entre los trabajadores. Se presentan varios casos de personas que hacen
	más labores que las que tienen asignadas, por ejemplo, trabajadoras del área de venta
	ayudan a limpiar y pulir algunas joyas ya fabricadas, haciendo más liviana la tarea
	asignada a los trabajadores del área de producción. Estas situaciones generan sin duda 
	un ambiente favorable.\\

%\subsection{Teoría de Fayol}
%	Usando los principios de Henry Fayol encontramos cierta correspondencia con la organización, por ejemplo
%	división de trabajo, autoridad en este caso representada por el Sr. González. Se nota disciplina y subordinación
%	del interés individual al interés global. El principio de unidad de mando no se cumple a cabalidad, debido a que
%	los trabajadores reciben órdenes tanto del señor Iván González padre como de Iván González hijo. No existen datos
%	sobre la equidad de la remuneración, pero si sabemos que todos comparten un contrato ``estandar''
%	para cada área de trabajo. La centralización es clara en la figura del jefe, respecto a contratación de 
%	personal es fácil notar que no existe mayor rotación como hicimos notar anteriormente.\\

\subsection{Teoría de Barnard}
	Analizando la organización desde el punto de vista de Barnard vemos que existe coordinación de varias personas
	para realizar las tareas diarias, tanto jefes como trabajadores, buscando el logro de los objetivos. Se ve
	que existe cierto fin común, que el producto sea de gusto del cliente.\\

	Es clara la idea de coordinación de varios trabajadores para realizar algunas labores, varios
	trabajadores mencionan que en los tiempos que tienen disponibles ayudan en otras labores, por ejemplo,
	limpieza de joyas, pulido, entre otras, todo esto enmarcado en la realización de trabajos de calidad.

\subsection{Teoría de Emery}
	La organización presenta relación e interés hacia la tecnología, y actualmente se encuentra en un
	proceso de transición del papel a lo digital. Como todavía no han implementado completamente el
	sistema ni capacitado a todo el personal, hay muchos empleados que no tienen mayor contacto con
	computadoras ni máquinas que agilicen los procesos productivos, manteniendo las mismas
	herramientas que han utilizado hace años, propias de su trabajo. Emery nos habla de sistemas sociales donde
	existe redundancia de funciones y roles flexibles, lo cual hemos logrado identificar en los
	sistemas sociales de los trabajadores de la empresa, a pesar de que todos tienen una labor, son
	capaces de hacer otra distinta y cooperar con los demas. Respecto a la disminución de la
	burocracia podemos notar que es posible que cada trabajador llegue facilmente al jefe, buscando
	apoyo ya sea laboral o de otra índole. Además como mencionamos anteriormente la organización
	entrega oportunidades de aprendizaje ya que la mayoría ha ido
	aprendiendo nuevas labores desde que llegó a la organización.\\

\subsection{Teoría de Durkheim}
	Logramos identificar un hecho independiente de las personas que encuestamos, el cual generaba un
	sentimiento ``familiar'' entre las personas pertenecientes a la organización. Esto puede ser identificado
	por un hecho social, generado por las políticas sociales de empresa. Esto generaba un ambiente de
	solidaridad y cooperación entre los distintos trabajadores, ayudando a la integración de los nuevos
	empleados con los que logramos conversar. Bajo este fenómeno social, podemos identificar que las
	distintas personalidades se veían influenciadas de diferente manera, presentandose más y menos
	fuertemente dependiendo del caracter de cada uno y el tiempo que llevaran trabajando en la
	organización.\\

