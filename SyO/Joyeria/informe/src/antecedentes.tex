%Investigación de las teorías a aplicar en la organización (cada una debe estar bien detallada).
%(No está permitido el copy/paste de las ppt del ramo).

\begin{itemize}
	\item \textbf{Luther\ Gullick:} \emph{``PODSCORB''}\\
	% Lista
	Basándose en el estudio de Fayol,
	postula apuntes de la Ciencia de Administracion (junto a Urwick),
	donde postula los 7 principios que un gerente debe realizar,
	``PODSCORB'',
	Además cabe mencionar que fue la teoría más importante del autor,
	la cual resumió en una sola palabra compuesta por las iniciales de cada uno de los siete principios:
	\begin{itemize}
		\item Planificación
		\item Organización
		\item Dirección
		\item Formación del plantel(Staffing)
		\item Coordinación
		\item Reporte
		\item Presupuestar
	\end{itemize}
	Gullick agrega nuevos elementos al proceso de la Administración,
	como lo es en el caso del ``Estado Benefactor'',
	ya que la crísis vivida en ese momento,
	requería mayor eficiencia en la ayuda que se daba en cuanto a demandas sociales,
	seguridad, alimentación, vivienda, entre otras.
	
	Por lo tanto,
	se dice que existe un estado benefactor cuando el estado asegura protección social.

	\item \textbf{Adam\ Smith:} \emph{``La Riqueza de las Naciones''}\\
	% Listo
	Adam Smith es el Padre de la economía moderna (capitalismo),
	en la teoría anteriormente señalada,
	Smith intentó diferenciar la Economía Política de la Ciencia Política,
	la ética y jurisprudencia.
	
	En ésta teoría critica al mercantilismo;
	búsqueda de un orden económico que funcionara mejor no tomando en cuenta el estado.

	Plantea también que la clave del bienestar social,
	está en el crecimiento económico y se potencia por la división del trabajo.

	``Gracias a la apelacion al Egoísmo se logra el bienestar General''

	Recordemos que en el Mercantilismo,
	el sistema se basa en la propiedad privada y
	en la utilización de los mercados como forma de organizar la actividad económica.
	es decir, maximizar el interés del Estado soberano y no de los propietarios.
	Lo cual era totalmente lo contrario del Capitalismo de Smith.
	
	Finalmente,
	Smith plantea dos ideas claves,
	para poder seguir el camino del Capitalismo:
	Explicar los factores que determinan el progreso económico y
	las medidas que podrían tomarse para crear ambiente favorable para el crecimiento económico y
	que los ingresos per capita se deben a la destreza de cada persona.
		
	\item \textbf{\'Emile Durkheim:} \emph{``Hechos sociales''}\\
	% Listo
	Es el Padre de la sociología;
	durante su vida planteó variadas ideas,
	como la del ``Trabajo Social'' y la idea de los ``Hechos sociales'',
	ésta última es la que se utilizará en nuestro trabajo y también
	es considerada la más importante.

	Primero que todo,
	debemos notar que  ``Los Hechos Sociales'' separaron la sociología de la sicología,
	señalando que todo lo que ocurría a nuestro alrededor,
	era un hecho social y poseía las siguientes características:
	\begin{itemize}
			\item Exterioridad, ocurren fuera de nuestra persona.
			\item Coerción, poseen cierta influencia sobre nosotros.
			\item Colectividad, no sólo nos ocurren a nosotros, sino a todos nuestros similares.
	\end{itemize}

	Los hechos sociales tienen otra condición no menos importante que las anteriores,
	y que es la de encarnarse en la psiquis de cada individuo de una sociedad
	y por tanto transformar la forma subjetiva de sentir determinados hechos o situaciones,
	por esta misma razón adquieren un carácter sui géneris,
	con valor en sí mismo y no como resultado de otros hechos sociales.
	
	Esta forma de sentir cuando el hecho se presenta frente a la presencia de un grupo puede dar
	lugar a otro fenómeno social, que Durkheim determina ``Corrientes Sociales''.

	Finalmente,
	lo que debemos comprender acá,
	es que según Durkheim
	todo hecho que ocurre a nuestro alrededor,
	es un hecho social (con las características anteriormente señaladas),
	es decir,
	la sociedad es lo más importante,
	y rige nuestro actuar, nuestro pensar, todo.
	
	\item \textbf{Frederick\ Taylor:} \emph{``Administración Científica''}
	
	La Administracion Científica de Taylor,
	aborda aspectos como estudios de tiempos y movimientos,
	selección de obreros,
	métodos de trabajo,
	incentivos,
	especialización e
	instrucción.

	La idea principal es optimizar la manera que el trabajo es realizado
	y simplificando el trabajo realizado por los trabajadores,
	para que fueran capacitados
	sólo para realizar esta función de la mejor forma posible.

	Esta teoría quitó esta autonomía y
	el trabajo que sólo un experimentado artesano
	ahora será realizado por trabajadores que tienen asignada una tarea simple de realizar
	y que es fácil de aprender. 

	Taylor identifica los males principales en una organización:
	\begin{itemize}
		\item Holgazanería por parte de los trabajadores. 
		\item Falta de información por parte de la Administración hacia los obreros.
		\item Falta de técnicas de trabajadores. 
	\end{itemize}

	Finalmente,
	Taylor al presentar su teoría se basa en los siguientes fundamentos:
	\begin{itemize}
		\item Desarrollar para cada elemento del trabajo del obrero,
			una ciencia que reemplaza los antiguos métodos empíricos. 
		\item Selecciona científicamente y luego instruye,
			enseña y forma al obrero de acuerdo con sus propias posibilidades. 
		\item Coopera cordialmente con los obreros,
			para que todo el trabajo sea hecho
			de acuerdo con los principios científicos que se aplican.
		\item Distribuye equitativamente el trabajo y
			la responsabilidad entre la administración y los obreros. 
	\end{itemize}

	\item \textbf{Mary\ Parker\ Follet:} \emph{``El Nuevo Estado''}\\
	% Listo
	Follet es una de las creadoras de los inicios básicos de la Administración.
	Su estudio sobresalió porque sostenía la existencia
	de principios generales,
	que eran aplicables en \emph{todas} las organizaciones.

	A través de su vida,
	defendio el principio de lo que ella llamó como \emph{``integración''};
	el ``poder con'' en vez del ``poder sobre''.
	
	Principalmente,
	la enseñanza que Follet nos deja,
	y que será utilizada en el análisis de ésta organización,
	es que los Jefes no deben tratar como cualquier cosa a los trabajadores,
	sino que cada trabajador es una persona y tiene sus derechos,
	por lo que deben sentirse parte de la organización,
	con buenos tratos.

	``Sólo podremos encontrar al verdadero hombre en la organizacion de grupo.''
	
	\item \textbf{Chester\ Barnard:} \emph{``Influencia de factores sicológicos y sociales en la efectividad de la organización''}\\
	% Listo
	Barnard nos entrega un esquema conceptual
	de la teoría de las organizaciones basado en:
	\begin{itemize}
		\item Racionalidad individual y grupal.
		\item Cooperación.
		\item Organizacion Formal e Informal.
	\end{itemize}

	Además el define la Organizacion, 
	como un sistema de actividades o fuerzas,
	conscientemente coordinadas de 2 o más personas.

	Otros puntos importantes en su estudio,
	es que avalaba el hecho de que los trabajadores de una organización,
	poseían limitaciones físicas y biológicas que los obligan a trabajar en grupo.
	Por lo cual la cooperación es fundamental en una organización.

	Su teoría ha servido para ir un paso mas allá
	en la Administración Científica y en los principios de Fayol.

	Finalmente,
	Barnard nos enseña que mantener una organización en un correcto funcionamiento implica:
	\begin{itemize}	
		\item Esforzarse para mantener la comunicación organizacional.
		\item Asegurar los servicios escenciales para los trabajadores.
		\item Formulación del ``propósito'' y de los objetivos.
	\end{itemize}

	\item \textbf{Fred\ Emery:} \emph{ ``Sistemas Sociotécnicos''}\\
	
	Emery nos plantea la teoría de los Sistemas Sociotécnicos
	en el ``Desarrollo Organizacional'',
	lo cual implica un esfuerzo planificado en toda la organización
	y controlada desde el nivel más alto de efectividad.

	A través de experimentos,
	logró darse cuenta que la tecnología sóla no funciona,
	por eso es necesario un sistema sociotécnico.

	Lo anterior se debe a que los trabajadores,
	al ver tecnología en la organización,
	se relaja, y comienza a haber mucho absentismo,
	además de haber ciertas disputas por lo mismo,
	crea un ambiente individualista y los trabajadores
	no se sienten parte de la empresa.

	Finalmente,
	el punto crucial en ésta teoría
	es que Emery se da cuenta que el sistema Tayloriano,
	es erróneo, por lo cual debemos hacer el trabajo
	en equipos, lo cual aparte de aumentar la productividad
	implica una cierta satisfacción personal,
	ya que se mejora mucho el ambiente de trabajo.
	
\end{itemize}
