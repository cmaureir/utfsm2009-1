\documentclass[spanish]{article}

\usepackage{amssymb}
\usepackage{amsmath}
\usepackage{amsfonts}
\usepackage{babel}
\usepackage[utf8]{inputenc}
\usepackage[hmargin=1cm,vmargin=1.5cm]{geometry}

\title{Weary of Looking for Work, Some Create Their Own}
\date{March 14th, 2009}

\begin{document}
\maketitle\thispagestyle{empty}
SAN FRANCISCO — Alex Andon, 24, a graduate of Duke University in biology,
was laid off from a biotech company last May. For months he sought new work.
Then, frustrated with the hunt, he turned to jellyfish.\\

In an apartment he shares here with six roommates, Mr. Andon started a business
in September building jellyfish aquariums, capitalizing on new technology that
helps the fragile creatures survive in captivity.
He has sold three tanks, one for \$25,000 to a restaurant, and is starting a Web
site to sell desktop versions for \$350.\\

“I keep getting stung,” he said. And his crowded home office is filled with beakers
and test tubes of jellyfish food. “But it beats looking for work. I hate looking for work.”\\

Plenty of other laid-off workers across the country, burned out by a merciless job market,
are building business plans instead of sending out résumés.
For these people, recession has become the mother of invention.\\

Economists say that when the economy takes a dive, it is common for people to turn to their
inner entrepreneur to try to make their own work. But they say that it takes months for
that mentality to sink in, and that this is about the time in the economic cycle when
it really starts to happen — when the formerly employed realize that traditional job searches
are not working, and that they are running out of time and money.\\

Mark V. Cannice, executive director of the entrepreneurship program at the University of San Francisco,
calls the phenomenon “forced entrepreneurship.”\\

“If there is a silver lining, the large-scale downsizing from major companies will release a lot
of new entrepreneurial talent and ideas — scientists, engineers, business folks now looking to do
other things,” Mr. Cannice said. “It’s a Darwinian unleashing of talent into the entrepreneurial ecosystem.”\\

Even in prosperous times, entrepreneurs have a daunting failure rate. But those who succeed could play a big
role in turning the economy around because tiny companies are actually big employers.
In 2008, 3.8 million companies had fewer than 10 workers, and they employed 12.4 million people,
or roughly 11 percent of the private sector work force, according to the Bureau of Labor Statistics.\\

Economists say there are some peculiarities to this wave of downturn start-ups. Chiefly,
the Internet has given people an extraordinary tool not just to market their ideas but also to find
business partners and suppliers, and to do all kinds of functions on the cheap: keeping the books,
interacting with customers, even turning a small idea into a big idea.\\

The goal for many entrepreneurs nowadays is not to create a company that will someday make billions
but to come up with an idea that will produce revenue quickly, said Jerome S. Engel, director for the
center for entrepreneurship at the Berkeley Haas School of Business.
Mr. Engel said many people will focus on serving immediate needs for individuals and businesses.\\

“It’s a very painful thing,” he said of the pressure people feel to find new ways to make money.
“But it’s a healthy thing.”\\

Mr. Engel noted that the dot-com bust helped propel a pack of hardy companies. One of those, in fact, was Google. While it was started in the late 1990s, the company succeeded during the bust in part because it was highly focused and did not need much capital, Mr. Engel said.\\

Ryan Kuder, 35, understands the notion of scaled-down start-up fervor — and the worry and exhilaration that goes along with it. He was laid off in February 2008 from Yahoo, where he was a senior marketing manager. He job-hunted for a bit, then decided to start an Internet company that would let people do social networking at the neighborhood level.\\

Mr. Kuder and his business partner toiled until November, when he realized his big dreams had run headlong into reality. He needed money to pay the mortgage and buy health insurance for his family.\\

They transformed the company into a new one called Koombea that designs and builds Web sites for businesses. Koombea has grown to nine people, most of them in Colombia, where the cost of living allows them to do Web design relatively inexpensively.\\

Mr. Kuder and his wife agreed that he would give up working for Koombea at the end of January if he did not hit certain revenue goals. They narrowly missed the target. For a few days, Mr. Kuder sent out résumés. He found no work, so he is back investing himself full time in Koombea — and says he is feeling transformed.\\

“My sleeves are rolled up, and I’m dirtier than I’ve ever been before,” he said. “It’s incredibly nerve-wracking. I wake up nauseous everyday. But it’s probably easier right now to find a problem, solve it and charge people than it is to find a job.”\\

Monica Zamiska, 25, said it was “traumatizing” when she was laid off in January from her first postcollege job as a junior account executive with the public relations firm Ogilvy \& Mather. After meeting with five recruiters, she began to realize how barren the job market was. “You can only send out so many résumés,” Ms. Zamiska said.\\

So she turned her full attention to a pet project called the Confoodant, a Web site with restaurant reviews written by a by-invitation-only network of food enthusiasts. Her main financial obligation is her rent, but with her savings and four weeks of severance pay, she is confident that she can devote at least six months to getting the project off the ground.\\

“I love working,” Ms. Zamiska said. “So I made work for myself.”\\

The surge of interest in entrepreneurship can be seen in the demand for related workshops and networking events. Monica Doss, director of FastTrac, an organization that offers training to aspiring entrepreneurs, said she expected participation to double this year from the 10,000 people it had last year.\\

“People are thinking, ‘These jobs aren’t going to come back in three years. I’ve got to find something else to do,’ ” Ms. Doss said.\\

Mr. Andon, for one, seems to have found his niche. He said he recently received an order for a large jellyfish tank that should sell for tens of thousands of dollars.\\

His entrepreneurial fever seems to be catching: at least four of his roommates are starting companies. Two of those — Erin Kitchell, 28, and her brother Andrew, 25 — are making laminated, fold-out language guides for travelers. In the next few days they plan to print their first 8,000 copies and start trying to sell them.\\

Ms. Kitchell took a voluntary buyout in June from Wachovia, sensing a layoff would come anyway, and is not sanguine about finding good work.\\

“This is as good a time as any to try something entrepreneurial,” she said. “There is not a lot of opportunity right now in finance.”\\



\end{document}
