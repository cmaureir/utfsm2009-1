\subsection{Contexto Filos\'ofico}

El pensamiento filosófico francés del siglo XIX y principios del siglo XX está dominado por el positivismo de Comte.
Pero a par de esta tendencia totalizadora, que termina degenerando en una grotesca parodia del culto católico, coexisten otras corrientes de cierta importancia:
así el espiritualismo de Alfredo Fouillée (1838-1912) y de Víctor Cousin (1892-1867) y el neocriticismo de Carlos Renouvier (1815-1903).
\begin{enumerate}
\item \textbf{Espiritualismo\ y\ Neocriticismo}

A Fouillée se debe la formulación del sistema de las "ideas-fuerzas", que consiste en considerar que la idea no es tan solo
una representación mental, sino un "principio que tiende a realizarse".
"Así, explica Reinach, en la controversia entre libre arbitrio y determinismo, concluye que ambas tesis se apoyan en argumentos
irrefutables, pero que el hombre posee la idea de la libertad y que esa idea le hace libre;
parejamente, la idea de justicia nos hace justos, y la idea de moral, que es un hecho de conciencia, nos lleva a la moralidad."

Cousin profesó una doctrina ecléctica de signo espiritualista.
"Los sistemas filosóficos, escribió, no son la filosofía: la filosofía los sobrepasa con toda la superioridad de un principio respecto de sus aplicaciones.
Los sistemas se esfuerzan por realizar la idea de la filosofía, así como las instituciones civiles se esfuerzan por realizar la idea de la justicia".
Cousin asumió siempre posiciones intermedias y conciliadoras, pero afirmando a la razón como árbitro supremo.

Renouvier, ya en su madurez, intentó conciliar los sistemas de Hume y de Kant, pero en realidad asumió una posición muy original.
Afirma que el nuómeno kantiano es "un vestigio de la metafísica escolástica";que lo continuo es una ilusión;
que la tesis de Hegel respecto a la identidad de los contrarios es insensata, y que los actos libres no son efectos sin causa, toda vez que su causa es el hombre mismo.
Por lo demás, consideró que si la naturaleza debe ser explicada matemática y mecánicamente, lo cierto es que el mecanismo no es otra cosa que la apariencia exterior
de esa misma naturaleza, en la que subyace esencialmente el pensamiento.

\item \textbf{Comte\ y\ el\ Positivismo}

El conocimiento humano, como la humanidad misma, se desarrolla en tres estadios: el teológico, el metafísico y el positivo.
En este último, la mente humana comprende que es imposible abarcar la esencia absoluta de la realidad y que únicamente se pueden establecer,
como explica Messer, los hechos y sus leyes, valiéndose para esto de la observación y de la experimentación y entendiéndose por "leyes" las
regularidades en el curso de los fenómenos.
Ahora bien: a diferencia de lo que ocurría en los estadios anteriores, en el positivo se llega a la conclusión de que todo conocimiento es relativo,
porque la realidad es avizorada desde el punto de vista del hombre y porque las leyes naturales nada nos dicen sobre "la esencia" de los fenómenos,
sino sobre sus relaciones.

Por lo que dice a la teoría del conocimiento, el positivismo comtiano es rigurosamente empirista.
"El Positivismo, escribe Preti, es decididamente laico, revolucionario y, por ello, antimetafísico. A las nebulosidades y arbitrios de un saber fundado en la fe, en el 'corazón', o en la intuición genial, contrapone el método objetivo, experimental o 'positivo' de la ciencia natural". | 101 Esto porque, dentro del ámbito espiritual europeo, el positivismo se presenta ante todo, como una reacción intelectual y racionalista contra el romanticismo y, por ende, contra el primado de la intuición, de la "sensibilidad" y de la "imaginación creadora".

\item \textbf{El\ Psicologismo\ Vitalista\ de\ Henri\ Bergson}

Henri Bergson fue, sin duda, el pensador más penetrante e influyente entre los franceses de su generación. Su carrera universitaria fue excepcionalmente brillante y sus escritos se distinguen por la claridad expositiva y las virtudes de un estilo literario de gran calidad.

El pensamiento de Bergson, ajeno al prurito sistemático, fluye de una posición esencialmente psicológica:la intuición directa de los datos inmediatos de la consciencia.
Para Bergson, uno de los primeros pasos en el camino de la filosofía consiste en distinguir el tiempo verdadero, que es el tiempo psicológico,
de su traducción especial, o sea del tiempo matemático.
Además, la vida al igual que la conciencia es duración, movilidad, renovada creación, libertad.
De aquí el que caractericemos su pensamiento bajo el epígrafe de "psicologismo vitalista".
\end{enumerate}
\newpage
