\subsection{Conclusiones Individuales}
%Conclusiones personales
Personalmente pude comprender la importancia de la sociedad en un individuo, sobre todo cuando Durkheim se refiere a que el suicidio es un hecho social y no un hecho individual,
la culpa no es del sujeto, es de la sociedad.

Otro aspecto importante es como la sociedad nos ense\~na a actuar, a pensar, todo respecto a sus mismos est\'andares, y la sociedad castiga muy duramente a quien no act\'ua como
ella misma le ense\~no, un aspecto muy claro que se puede apreciar mirando una cultura determinada, el extremismo y como muchas personas viven con sus ojos vendados, aceptando
una verdad local, que a los ojos del resto del mundo puede estar erroneo, pero ?`No es el resto del mundo otra macro-cultura que también nos obliga a pensar de una forma?

Principalmente, pude darme cuenta que la visi\'on que ten\'ia Durkheim con respecto a la sociedad, fue una marca intachable en la historia del pensamiento humano,
ya que el separandose de todos los aspectos posibles, dejando atr\'as emociones, prejuicios, creencias, pudo generar un juicio imparcial para poder formar o declarar
la sociolog\'ia actual como una ciencia, que hoy en d\'ia es una de las ciencias mas importantes para el estudio de la humanidad

Finalmente, a nivel  personal me llamo mucho la atenci\'on como pudo tener la valent\'ia en su tiempo de poder analizar fen\'omenos sociales, sin ninguna influencia tan notoria
dejando de lado pensamiento que ven\'ian de 8 generaciones antes que el a nivel familiar, y que si nos damos cuenta claramente son la estructura de la organizaci\'on llamada sociedad.
Sin dejar de lado, la capacidad que tuvo para poder describir que una organizaci\'on se puede llevar a la autodestrucci\'on (Anomia), t\'ermino que sirvi\'o y sirve notablemente al
momento de analizar los riesgos de las organizaciones actuales.



\subsection{Conclusiones Generales}
%Conclusiones 'en conjunto' ......

La visi\'on de Durkheim de la sociedad marc\'o al pensamiento mundial en una manera que era
muy necesaria, dejar de lado todo tipo de prejuicios y emociones para generar un juicio imparcial y
racional de \'esta se volvio el pilar de la Sociolog\'ia, ciencia que hoy en dia se ha establecido como
una de las mas importantes de las Ciencias Sociales.

A trav\'es de las ideas del autor, las empresas se pueden analizar con ``sangre fr\'ia'', extrapolan-
do las ideas de ``hechos sociales'' y ``estructura'' a organizaciones mas pequeñas. De esta manera se
puede entender mejor la forma de comportarse de algunas organizaciones que tienden
a la autodestrucci\'on o que logran surgir en ambientes altamente hostiles.

%	\item Sobre los hechos sociales
Adem\'as, cuando uno es insertado en una nueva comunidad, como lo es una organizaci\'on o empresa hoy en d\'ia, uno 
se ve rodeado de un sistema con una cultura propia y muchas veces diferente a la del individuo. Estos 
detalles impl\'icitos muchas veces dentro dela organizaci\'on, son los reconocidos como ``hechos sociales''
por Durkheim. Por ello, podemos concluir que fue importante que Durkheim halla reconocido que los 
hechos sociales deb\'ian ser analisados por m\'etodos diferentes a los de la psicolog\'ia, ya que los estudios 
de sociolog\'ia desarrollados por \'el han sido un gran aporte para el estudio de la administraci\'on general 
de estos ``sistemas humanos'', jugando un rol medular en el asunto. 

%	\item Sobre las Corrientes Sociales\\
Tambi\'en, esto dicho anteriormente  deber\'iamos encararlo con objetividad, desprendiéndonos de todos los prejuicios y preconceptos que podamos tener antes de abordarlos.\\
Esto puede llegar a ser muy difícil, si a modo de ejemplo tomamos por punto de partida que el analista pertenece a una colectividad, a una sociedad, que tiene determinado su pensamiento a través del lenguaje que determina en sí mismo una estructura preestablecida de pensamiento lógico.


%	\item Sobre la divisi\'on del trabajo social\\
Para el estudio y administraci\'on de empresas u organizaciones, el concepto de solidaridad desarrollado
por Durkheim es importante de recalcar. Vi\'endolo desde su punto de vista, cada organizaci\'on deber\'ia 
buscar una solidaridad mec\'anica entre sus miembros. De esta forma, los individuos lograr\'ian adquirir 
una mentalidad enfocada a los objetivos de la empresa (vista como comunidad), y as\'i, mantener un 
ambiente de unidad y confianza en torno a los ideales de \'esta. Es notable mencionar que muchas 
organizaciones tienden a de hoy en dia tienden a fomentar un ambiente competitivo entre sus integrantes, 
lo cual, al igual que la mentalidad individualista en la realidad actual, conlleva muchas veces a problemas 
t\'ipicos de una solidaridad org\'anica.

%	\item  Sobre la educaci\'on\\
Ahora si analizamos las las funciones de la educaci\'on destacadas por Durkheim, desde el punto de vista de las organizaciones, 
la educaci\'on sirve como herramienta para lograr un mayor orden social, una mayor lealtad y una mayor especializaci\'on 
de los integrantes hacia la comunidad, siendo as\'i de gran importancia para mejorar la integraci\'on de nuevos miembros
y el crecimiento de la empresa.
	
%	\item  Sobre el Crimen
%Durkheim identific\'o al crimen como un indicador de la necesidad de cambios dentro de las organizaciones. 

%	\item Sobre la Ley
%Postulados principales:
% Diferencia duramente entre el Individualismo y la avaricia y el egoismo
% Avaricia y egoismo no son en lo absoluto posturas morales
% Estudia la Ley como una expresi\'on que garantizaba los valores fundamentales de la sociedad.

%	\item Sobre el Suicidio\\
De la misma manera, Durkheim, con su trabajo sobre el suicidio de los individuos con diferentes hechos sociales, logra destacar como ciertos 
factores en la sociedad (debilitamiento de los lazos de integraci\'on, desregularizaci\'on moral y falta de definiciones 
leg\'itimas, sociedades muy opresivas) pueden llevar a la autodestrucci\'on de sus miembros.\\
Por ello, se puede concluir que  es importante que las organizaciones se preocupen de mantener a sus integrantes 
cuidando de no desarrollar estas fallas que requieren de una reconstrucci\'on social.

%	\item Sobre la Religi\'on\\
Por \'ultimo, sus estudios sobre la religi\'on permitieron comprender mejor el desarrollo del orden social en las primeras comunidades, 
cumpliendo estas la funci\'on de separar lo mundano o cotidiano de las cosas intangibles, que relacionadas con lo sagrado, 
lograban la generaci\'on de normas que fomentaban el orden social.



\newpage
