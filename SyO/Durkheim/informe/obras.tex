El conjunto de trabajos de su obra la podemos resumir en siete puntos básicos:
\subsubsection{La solidaridad social}
		``La División del Trabajo Social'' (editada en 1893) fue su primer trabajo importante.
		La misma nació como la tesis doctoral con la que se recibió: ``La Solidaridad Social''.
		En ella intenta explicar la sociedad moderna mediante la división del trabajo y el derecho
		represivo por un lado, y por otro establece la crítica de la misma estableciendo la relación
		deseable entre el conocimiento positivo y el juicio normativo.
\subsubsection{El afincamiento de la sociología como ciencia autónoma}
		En dicho tópico sus obras fundamentales son: ``Las Reglas del Método Sociológico'' (1895) y ``El Suicidio'' (1897).
		En la primera define los principios epistemológicos de una ciencia positiva capaz de abordar al conocimiento
		concreto de las sociedades humanas, en forma totalmente independiente de las demás ciencias, esto es la sociología
		como ciencia autónoma; cosa que aún no habían podido definir ni Comte ni Spencer.
		En el segundo, realiza un estudio sociológico donde demuestra que lo que aparenta ser un hecho individual no es otra
		cosa que un hecho social, donde se relaciona la dependencia del individuo a factores externos y colectivos como son
		la religión, la economía y la familia.-
\subsubsection{Educación y pedagogía}
		Su artículo ``Educación'' publicado en el Nuevo Diccionario de Pedagogía y de Instrucción Primaria (1911),
		constituye un resumen de su pensamiento pedagógico. También dictó cursos en su cátedra sobre educación moral,
		historia de la pedagogía, éstos en las universidades de Burdeos y en la de París.
\subsubsection{Teoría política y derecho}
		Parte de la filosofía económica, jurídica y política del siglo XVIII y en base al estudio que hiciera de la obra
		de Saint Simón toma una posición eminentemente crítica respecto a las corrientes socialista y comunista.\\
		En la ``Física de las Costumbres y del Derecho'' (obra póstuma editada en 1950), compilación de sus cursos dictados,
		se divide el tema en dos partes: las solidaridades del grupo (la moral profesional y la moral cívica) y las solidaridades
		universales (donde trata sobre el respeto a la vida y al derecho de propiedad).
\subsubsection{La moral}
		Fue éste un tema recurrente en toda su obra:
		\begin{itemize}
			\item ``La Ciencia de la Moral en Alemania'' (1887)
			\item ``La Determinación del Hecho Moral'' (1906)
			\item ``Juicios de Valor y Juicios de Realidad'' (1911)
		\end{itemize}
		fueron los avances de la obra que no pudo editar en vida: ``Introducción a la Moral'';
		esta obra se canaliza en tres grandes temas:
		\begin{itemize}
			\item \textbf{a} concepto de la moral
			\item \textbf{b} el papel del moralista
			\item \textbf{c} desarrollo del concepto de una ciencia moral adecuada a sus tiempos.
		\end{itemize}
\subsubsection{La filosofía}
		Dada su formación filosófica, encara a la sociología con este perfil. Desarrolla una teoría sociológica de carácter
		ontológico en su obra ``Las Formas Elementales de la Vida Religiosa'' (1912).
\subsubsection{La religión}
		De su educación familiar en la tradición judía y su fe en la religión de la humanidad , se desprende la obra citada
		precedentemente en el item anterior.\\
		\emph{"La religión consiste en creencias y en prácticas relativas a las cosas sagradas".}\\
		Su concepto básico, en este tema, radica en comprender lo religioso en relación con lo sagrado sin necesidad de interponer
		los conceptos de la divinidad y el mas allá.
		Le importa demostrar que la experiencia religiosa no es exclusividad de sociedad alguna en particular, sino que por el
		contrario es un fenómeno universal.
		Entiende, por ser que históricamente todas las sociedades han experimentado un sentimiento religioso, que resulta
		imprescindible explicar la religión como un hecho social. La entiende como una experiencia real y no un acto imaginativo
		ya que es causa objetiva, universal y eterna de la religión de la humanidad.
		En resumen la misión de la ciencia social al respecto, es la de investigar el porqué de la causa de la religión como hecho
		social y no así el cuestionamiento de la religión en sí.
\newpage

