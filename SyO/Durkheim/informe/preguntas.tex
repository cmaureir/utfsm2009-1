\begin{itemize}
	\item ?`Cu\'al es la teor\'ia o postulado m\'as importante del autor?\\
	El postulado m\'as importante de Durkheim es el de los ``hechos sociales'' en el cual identifica que existen hechos externos y anteriores a las personas
	que s\'olo pueden ser explicados desde el punto de vista de la sociedad en que \'estas viven. Con \'esto, Durkheim separ\'o el estudio de la sociedad de la rama 
	de la psicolog\'ia y form\'o la rama de sociolog\'ia moderna reconocida hasta hoy en d\'ia.

	\item ?`En qu\'e forma influye en las empresas (organizaciones) de su \'epoca y actualmente esta teor\'ia?\\
	Principalmente en la forma de afrontar ciertos fen\'omenos sociales, ya que \'estos anteriormente eran estudiados bas\'andose en los estudios de la psicolog\'ia.
	Gracias a la separaci\'on y las bases m\'as cient\'ificas utilizadas por Durkheim para describir y estudiar este tipo de fen\'omenos, es se han podido desarrollar
	las teor\'ias sobre las sociedades y como distintos factores de \'estas influyen positiva o negativamente en las personas que la conforman.

	\item ?`Qu\'e aspectos de la teor\'ia son dif\'iciles de comprender o aplicar?\\
	Sobre su teor\'ia de la ``solidaridad social'', un t\'ermino acu\~nado por Durkeim en el cual identifica un tipo de solidaridad en cada sociedad que influye en los 
	niveles de integraci\'on social de las personas, y induce un tipo de conciencia espec\'ifica a los individuos de cada comunidad. Al ser un t\'ermino usado para 
	explicar un ``comportamiento'' abstraido de los ``hechos sociales'' de la sociedad, es dif\'icil de comprender en un comienzo, pero no deja de ser un punto muy 
	importante en su teor\'ia sobre los fenomenos sociales. 

	\item ?`Qu\'e aspectos de la teor\'ia son criticados negativamente y cuales son alabados positivamente en la literatura? ?`Cu\'al es su posici\'on al respecto?\\
	De todos los estudios desarrollados por Durkheim, el \'unico que fue duramente criticado fue su trabajo sobre ``El Suicidio''. En \'este aspecto, sus concluciones sacadas del 
	estudio estad\'istico desarrollado (relacionando la mayor cantidad de suicidios en los de religi\'on Protestante en comparaci\'on con los de religi\'on Cat\'olica) 
	parecian estar limitadas a s\'olo el grupo estudiado, lo cual, hacia dudar de las generalizaciones realizadas en las concluciones alcanzadas por Durkheim. A pesar de ello,
	\'estos estudios han servido de inspiraci\'on para muchos pensadores y estudiosos de la Sociolog\'ia. Adem\'as, aunque uno pueda discutir el nivel de verdad de lo desarrollado 
	en lo respectivo a los suicidios, las conclusiones alcanzadas pueden extenderse y aplicarse de forma objetiva en otras situaciones sin muchas dificultades, por lo que 
	apoyamos a los conceptos e ideas obtenidas, a pesar de que el m\'etodo desarrollado para obtenerlas talvez no halla sido el m\'as adecuado.

	\item ?`Qu\'e lecciones aprende de su an\'alisis?\\
	La importancia de la sociolog\'ia en la sociedad moderna, partiendo de la base de que el an\'alisis de ciertos hechos o influencias en nuestra vida, como lo son,
	la religi\'on, el crimen, la educaci\'on, el suicidio, etc, pueden dar pie a una cierta estructura la cual seguimos, por ende gracias a las lecturas realizadas,
	nos pudimos dar cuenta como \emph{todo} factor externo, nos influencia a ser como somos, hasta el mas m\'inimo acto de algun cercano, va restringiendo nuestro
	actuar frente a cualquier fen\'onmeno. Todo depende de la sociedad, y para estudiar dichos fen\'omenos necesitamos s\'olo t\'ecnicas sociales, ning\'un otro
	mecanismo sirve.

	\item ?`C\'omo la vida personal del autor influy\'o en sus teor\'ias?\\
	Su vida en general, estuvo siempre ligada a la realidad de la comunidad judia en Francia, s\'olo un a\~no abandon\'o su pa\'is para ir a estudiar en Alemania, pero
	volvi\'o a la brevedad. Parti\'o sus estudios en Filosof\'ia e Historia y de a poco fue acerc\'andose a la Pscicolog\'ia, campo en el cual se di\'o cuenta de la
	importancia de la Sociolog\'ia, raz\'on por la cual deb\'ia ser una ciencia distinta. Junto a algunos colegas fundo una revista dedicada a su estudio la cual tuvo
	mucho \'exito, por ende su vida, estuvo siempre ligadas a sus estudios, ya que a medida que iba abarcando nuevos estudios iba haciendo nuevas investigaciones.
	Ya que se sent\'ia influenciado por el m\'etodo cient\'ifico, y ten\'ia algunas creencias cat\'olicas, pod\'ia realizar estudios abarcandolos de los dos puntos de
	vista, por ende su vida, y el ambiente de su crecimiento (reinado por el positivismo) fueron dando pauta a el esqueleto de sus estudios.

	\item ?`Qu\'e c\'odigo de conducta o creencias marcaron la forma de pensar del autor analizado?\\
	Durkheim proven\'ia de una familia judia, ocho generaciones de rabinos fueron con su padre y el quebr\'o dicha tradici\'on, ya que por influencias de un profesor de
	la universidad se convirtió al catolicismo. Por \'este motivo y pensando en que la gente criticaria su forma de actuar y de pensar aludiendo a un \emph{``es por sus
	tradiciones religiosas que posee dicha opini\'on''}, decidi\'o alejar sus pensamientos y trabajos lo m\'as posiblemente de la religi\'on y tratar de dejar de lado
	todo el espiritualismo y tradiciones. De todas formas en m\'as de alg\'un trabajo critica fuertemente la religiosidad, comparandola con una especie de sociedad primitiva y buscando similitudes entre los distintos tipos de creencias de las diferentes religiones.

	\item ?`Qu\'e valores o principios \'eticos muestran los escritos del autor analizado?\\
	Claramente, sus estudios le dan gran importancia a los principios del orden, la responsabilidad social y la solidaridad entre los miembros de una sociedad. Marcado de 
	una u otra forma por sus antecedentes religiosos, Durkheim fundo sus conceptos bajo la idea que la sociedad debe tender a un orden social que logre la integraci\'on de 
	sus miembros orientados hacia el bien com\'un de la comunidad. Estos principios son los principales por los cuales se rigen las llamadas ``sociedades primitivas'', tales 
	como los sistemas organizacionales de la Iglesia o el Ejercito. A\'un as\'i, logro destacar la importancia del valor \'etico-social del individualismo (lo cual no debe 
	confundirse con la avaricia y el egoismo, los cuales no son en lo absoluto considerados como posturas morales por Durkheim).
	Durkheim vi\'o al individualismo como la base de los derechos humanos y los valores de la dignidad individual humana y su individualidad aut\'onoma, siendo muy importantes 
	para el desarrollo de las sociedades modernas caracterizadas por solidaridad org\'anica.

\end{itemize}
\newpage
