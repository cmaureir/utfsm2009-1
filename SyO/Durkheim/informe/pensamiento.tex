
\subsubsection{Los Hechos Sociales}
Sus antecesores tanto Comte como Spencer ve\'ian esta nueva ciencia con un profundo esp\'iritu positivista, d\'andole cualidades meramente organicistas o psicol\'ogicas, en tanto Durkheim epistemol\'ogicamente la independiz\'o de las restantes disciplinas cient\'ificas existentes.
 Es entonces que interpreta la existencia de fen\'omenos espec\'ificamente sociales a los que llam\'a ``hechos sociales'', que constituyen unidades de estudio que no pueden ser abordados con otras t\'ecnicas que no sean las espec\'ificamente sociales.\\
Durkheim define a los hechos sociales como: ``modos de actuar, de pensar y de sentir exteriores al individuo, y que poseen un poder de coerci\'on en virtud del cual se imponen''.\\

Las caracter\'isticas b\'asicas que representan a los hechos sociales son:
\begin{itemize}
	\item Exterioridad
	\item Coerci\'on
	\item Colectividad
\end{itemize}

Los hechos sociales existen con anterioridad al nacimiento de un individuo en determinada sociedad, por lo tanto son exteriores a \'el.\\
Por formar parte de la cultura de una sociedad son colectivos.\\
Y siendo que un individuo es educado conforme a las normas y reglas que rigen la sociedad donde naci\'o, son coercitivos.\\
Durkheim mismo ejemplifica hechos sociales genuinos diciendo: ``... si exist\'ian antes es que existen fuera de nosotros. El sistema de signos que utilizo para expresar mi pensamiento (lengua materna), el sistema monetario que empleo para pagar mis deudas, ...''\\
Analizando estos ejemplos llegamos a la conclusi\'on que todo rol que desempe\~namos en nuestra relaci\'on con los dem\'as seres humanos est\'an comprendidos dentro de un hecho social.\\

Sobre la coerci\'on, vale hacer una lectura de lo que el mismo dice: ```...Estos tipos de conducta o de pensamiento no son s\'olo exteriores al individuo, sino que est\'an dotados de un poder imperativo y coercitivo en virtud del cual se imponen a \'el, lo quiera o no.- ``... La conciencia pública reprime todo acto que la ofende''\\
``...Si yo no me someto a las convenciones del mundo, si al vestirme no tengo en cuenta los usos vigentes dentro de mi país y de mi clase, la risa que provoco, el alejamiento en que se me mantiene, producen, aunque en forma mas atenuada, los mismos efectos que un castigo propiamente dicho.''\\
Sobre la característica de la colectividad sigue diciendo: ``...Lo que los constituye son las creencias, las tendencias, las prácticas del grupo considerado colectivamente''. A título de ejemplo basta citar las normas parentales del antiguo pueblo judío que exigía ante la viudez de una cuñada, que el cuñado estaba comprometido a tomarla como esposa, también, y la protegería; iguales ejemplos son los diferentes conceptos actuales de diversas sociedades en torno a la relación marital, o bien a la primacía de un sexo sobre el otro en el contexto y comportamiento social.\\
Los hechos sociales tienen otra condición no menos importante que las anteriores y que es la de encarnarse en la psiquis de cada individuo de una sociedad y por tanto transformar la forma subjetiva de sentir determinados hechos o situaciones, por esta misma razón adquieren un carácter sui géneris, con valor en sí mismo y no como resultado de otros hechos sociales.\\
Esta forma de sentir cuando el hecho se presenta frente a la presencia de un grupo puede dar lugar a otro fenómeno social, el que pasamos a describir.\\

\subsubsection{Las Corrientes Sociales}
En la obra anteriormente mencionada, Durkheim los describe de la siguiente manera: ``...Así en una asamblea, los grandes movimientos de entusiasmo, de indignación, de piedad que se producen, no tienen como lugar de origen ninguna conciencia particular. Nos llegan a cada uno de nosotros desde fuera y son susceptibles de arrastrarnos a pesar nuestro. …Si un individuo intenta oponerse a una de esas manifestaciones colectivas, los sentimientos que rechazan se vuelven en su contra.''\\
Estas situaciones suelen ser de carácter emocional y por tanto breves, en algunas condiciones toman un giro racional, transformando así su permanencia, con lo que pueden volverse duraderas. Esto se puede demostrar por las diferentes etapas históricas por la que pasa un país cuando es marcado por un hito particular, por ejemplo el nacimiento de los partidos políticos tradicionales en el Uruguay.\\

\subsubsection{La divisi\'on del Trabajo Social}
Para esta obra Durkheim parte de la base del concepto de solidaridad.\\
Opone la organización de las sociedades arcaicas frente a la moderna y en relación con el espacio productivo que posean para su desarrollo.\\
En las sociedades pequeñas numéricamente y con amplia extensión productiva, la división del trabajo es casi imperceptible. A modo de ejemplo, en las civilizaciones neolíticas la caza era una función masculina, mientras que el laboreo de la tierra era de carácter femenino.\\
Estas sociedades por estar constituidas por segmentos sociales iguales les da el carácter de ``segmentado'' y en ellas existe un principio de ``solidaridad mecánica''.\\
Como consecuencia la solidaridad de la colectividad es muy estrecha por lo que la conciencia colectiva prácticamente anula a la individual. En estas sociedades, incluso la religión es unificadora.\\
Por el contrario en la medida que la sociedad crece numéricamente se hace imprescindible la diversificación del trabajo para poder atender las necesidades de la colectividad.\\
Esta diversificación laboral estratifica a la sociedad acorde a sus funciones, y en este sistema se establece lo que Durkheim da en llamar una ``solidaridad orgánica''.\\
Por el hecho que en las sociedades organizadas los individuos desarrollan diferentes aptitudes, aquellos que se concentran en un mismo tipo de funciones desarrollan diferentes enfoques de pensamiento, de estética, de ética, etc., por lo que la conciencia individual de un grupo se diferencia de los otros, y a su vez lo mismo ocurre con el individualismo dentro de cada subgrupo social.\\
Ante esta situación de crecimiento social, Durkheim, establece el concepto de ``densidad moral'' o ``dinámica''.\\
Dice: ``...cuanto mas numerosos son (los individuos) y cuanto más intensa es la acción de unos sobre otros, tanto más reaccionan con fuerza y rapidez y por consiguiente, tanto más intensa es la vida social''.\\
Con lo que la diversificación del trabajo es la solución encontrada, por él, ante la escasez, producto del crecimiento demográfico en un mismo espacio.\\
Mas adelante profundizará: ``...La división del trabajo varía en razón directa al volumen y a la densidad de las sociedades, …''.\\
Resumiendo, Durkheim especifica únicamente, que el crecimiento demográfico es la causa de todos los demás cambios sociales, por lo que su teoría en este tópico, ha sido clasificada como ``reduccionista''.\\
Sin embargo, en función del análisis que él hace sobre las sociedades de China y Rusia de su época, daría a entender que el crecimiento demográfico sería la causa de una mayor ''densidad dinámica``\\

\subsubsection{Educaci\'on}
Durkheim tambi\'en estaba interesado en la educaci\'on.

 Esto fue en parte porque trabajaba profesionalmente como capacitador de docentes, y \'el utiliz\'o su capacidad para afinar su curriculum para alcanzar sus metas de ense\~nar sociolog\'ia y expandirla como ciencia propia.

En t\'erminos m\'as generales, sin embargo, Durkheim estaba interesado en la forma en que la educaci\'on podr\'ia ser usada para proporcionar a los ciudadanos franceses la clase de contexto de solidaridad y secularidad necesaria para prevenir la anomia en la sociedad moderna.

Es en este sentido que tambi\'en propuso la formaci\'on de grupos de profesionales para servir como una fuente de solidaridad para los adultos. \\

Durkheim sostuvo que la educaci\'on tiene muchas funciones:

\begin{enumerate}
	\item Mejorar la solidaridad social
	\begin{itemize}
		\item Historia: Aprender sobre como las personas que han hecho buenas obras por los dem\'as hace sentir insignificante al individuo. 
	 	
% pledging -> ''promesas de contribuciones'', nose bien como traducirla, lo deje como:
          	\item Lealtad: Hace que las personas se sientan parte de un grupo y, por tanto, tiendan a haber menos posibilidades de que rompan las reglas.
	\end{itemize}
	\item Mantener el orden social
	\begin{itemize}
		\item La escuela es una sociedad en miniatura. Tiene una jerarquizaci\'on, reglas y espectaciones similares al del ``mundo exterior''. \'Esta entrena a la gente joven para que cumpla sus roles.
	\end{itemize}
	\item Mantener la divisi\'on del trabajo
	\begin{itemize}
		\item La escuela ordena a los estudiantes por grupos segun sus habilidades, animandolos a que elijan trabajos seg\'un las habilidades de cada uno.
	\end{itemize}
\end{enumerate}

\subsubsection{Crimen}
Durkheim ve\'ia que el crimen era el abandono de las nociones convencionales.
\'El creia que el crimen estaba ``ligado a las condiciones fundamentales de toda vida social'' y cumple una funci\'on en la sociedad.
\'El dictaba que el crimen era ``un util preludio para las reformas''.\\
En este sentido, vio al crimen capas de soltar ciertas tensiones sociales, teniendo un efecto purificador en la sociedad.
Llendo m\'as all\'a, estato que la autoridad que la conciencia moral disfrute no debe ser excesiva, ya que nadie se atrever\'ia a criticarla, provocando f\'acilmente un estado de forma inmutable.
Para que el progreso exista, la originalidad debe ser capaz de expresarse a s\'i misma, por lo que la originalidad del criminal tambi\'en debiera posibilitarse.

\subsubsection{Ley}
M\'as alla del estudio espec\'ifico de la ley, el crimen y el orden, Durkheim estaba profundamente interesado en el estudio de la ley y sus efectos sociales en general.
Entre los teoricos de la sociolog\'ia cl\'asica, el es uno de los fundadores de la socilogia de la ley.\\
En sus trabajos, el vio tipos de leyes, distingidas como represivas versus restitutivas (caracterizadas por sus sanciones), como reflejo directo de los tipos de la solidaridad social.
Por ello, el estudio de la ley fue de interes a la socilog\'ia por el hecho de poder revelar detalles sobre la naturaleza de la solidaridad.
Sin embargo, luego, \'el enfatiz\'o la importancia de la ley en s\'i como campo de estudio de la sociolog\'ia.
M\'as tarde, la visi\'on Durkheimiana vio a la ley (civil y criminal) como una expresi\'on que garantizaba los valores fundamentales de la sociedad.
Durkheim empatiz\'o con la forma en que la ley moderna crecientemente expresa una forma moral individualista - un sistema de valores que es, desde su punto de vista, probablemente el \'unico universalmente apropiadado a las condiciones modernas de la solidaridad social.
Individualismo, en este sentido, es la base de los derechos humanos y los valores de la dignidad individual humana y su individualidad autonoma.
Es duramente distingida de la avaricia y el egoismo, los cuales para Durkheim no son en lo absoluto posturas morales.\\
Muchos de sus seguidores m\'as cercanos, como Marcel Mauss, Georges Davy, Paul Fauconnet, Paul Huvelin, Emmanuel Levy y Henri Levy-Bruhl tambi\'en se especializaron o contribulleron al estudio de ley en la sociolog\'ia.

\subsubsection{El Suicidio}
En ``Suicide'' (1897), Durkheim explora las diferentes tasas de suicidios entre Protestantes y Cat\'olicos, concluyento que el mayor control social entre los Cat\'olicos resultaba en menores tasas de suicidio.
De acuerdo con Durkheim, la socidad Cat\'olica poseia niveles normales de integraci\'on social mientras que los Protestantes tenian niveles m\'as bajos.
Hay al menos dos problemas con \'esta interpretaci\'on.
Primero, Durkheim tom\'o la mayor\'ia de sus datos de investigadores anteriores, probablemente de Adolf Wagner y Henry Morselli, los cuales fueron mucho m\'as cuidadosos con la generalizaci\'on a partir de sus propios datos.  
Segundo, otros investigadores encontraron que las diferencias de las tasas de suicidios parecian estar limitadas a los Germano hablantes de Europa, por lo que podria haber sido el reflejo de otros factores.
A pesar de sus limitaciones, su trabajo en el suicidio a influenciado diferentes exponentes de la teor\'ia del control mencionados en el estudio de la sociolog\'ia cl\'asica.\\
Las conclusiones sobre el comportamiento individual son basados en las estadisticas de los suicidios. Este tipo de inferencias pueden ser a veces enga\~nosas, como son mostradas por los ejemplos dela paradoja de Simpson.\\

Durkheim dict\'o que habian cuatro tipos de suicidio.
\begin{description}
	\item[Suicidios Egoistas:]
		Son el resultado del debilitamiento de los lazos de integraci\'on de los individuos con la colectividad. En otras palabras, el quiebre de la integraci\'on social.\\
		Es la falla sintom\'atica del desarrollo econ\'omico y la divisi\'on del trabajo para producir la solidaridad organica de Durkheim.\\
		Su remedio esta en una recnostrucci\'on social.
	\item[Suicidios Altruistas:]
		Esto ocurre en sociedades con un alto nivel de integracion, donde las necesidades del individuo son menos importantes que las necesidades de la sociedad.\\
		Como el interes individual no es importante, Durkheim estima que en una sociedad altruista habrian pocas razones por las cuales las personas cometieran suicidio.\\
		Solo di\'o una exepci\'on: si el individuo espera que el suicidarse ayude a la sociedad.
	\item[Suicidios Animos:]
		Son el producto de la desregularizaci\'on moral y la falta de definiciones de aspiraciones leg\'itimas, dadas por una \'etica social restrictiva, la cual puede imponer orden y definir la conciencia individual.\\
		Es la falla sintom\'atica del desarrollo econ\'omico y la divisi\'on del trabajo para producir la solidaridad organica de Durkheim.\\
		Las personas desconocen donde ellos calzan dentro de su sociedad. \\
		Su remedio esta en una recnostrucci\'on social.
	\item[Suicidios Fatalistas:]
		Este tipo de suicidio parece ocurrrir en sociedades muy opresivas, causando que la gente prefiera morir que vivir dentro de la sociedad.\\
		Esta es una raz\'on extremadamente rara para que alguien se quite la vida, pero un buen ejemplo puede ser ir a prisi\'on; las personas pueden preferir la muerte que vivir en una prisi\'on con constantes abusos.
\end{description}

\subsubsection{Religi\'on}
En la socilog\'ia cl\'asica, el estudio de la religi\'on se concentraba principalmente en dos interrogantes:
\begin{enumerate}
	\item C\'omo la religi\'on contribuye a mantener un orden social? 
	\item Cual es la relaci\'on entre la religi\'on y la sociedad capitalista?
\end{enumerate}
Estas dos preguntas fueron t\'ipicamente combinadas con el argumento de que el capitalismo industrial cocavar\'ia los hechos religiosos tradicionales y por ello amenazar\'ia a la cohesi\'on de la sociedad.
Mas recientemente, el tema se a enfocado en el estudio de las instituciones religiosas. \\
En su art\'iculo,``El origen de las Creencias'' Durkheim se puso a si mismo en la tradici\'on positivista, significando que el pens\'o su estudio de la sociedad como desapasionado y cient\'ifico.
\'El estaba profundamente interesado en el problema que ten\'ian las sociedades modernas.
La religi\'on, argumentaba, era una expresi\'on de cohesi\'on social. \\

Sus intereses subyacentes eran el entender la existencia de una religi\'on en la ausencia de creencias de sus principios.
Durkheim veia el toteismo como la forma m\'as basica de religi\'on.
Es en este sistema de creencias en donde la separaci\'on entre lo sagrado y lo profano se ve claramente.
Todas las otras religiones, dice, son derivaciones de esta distinci\'on, agregandole mitos, iconos y tradiciones.
El animal tot\'emico, cre\'ia Durkheim, era la expresi\'on de lo sagrado y el foco original de la actividad religiosa a causa de que era el emblema del grupo social, del clan.
As\'i la religi\'on es inevitable, de igual forma como es inevitable que los individuas vivan juntos como grupo en una sociedad.\\

Durkheim penso que ese modelo para las relaciones entre las personas y lo sobrenatural fue la relaci\'on entre los individuos y la comunidad.
Pensaba que las personas ordenaban el mundo f\'isico, supernatural y el social acorde a principios similares.\\
Su primer prop\'osito era el identificar el origen social de la religi\'on, ya que la veia como una fuente de camaderias y solidaridad. 
Era una forma en que los individuos se volv\'ian reconocibles en frente de una sociedad establecida.
Su segundo proposito era el de identificar v\'inculo entre ciertas religiones en diferentes culturas, encontrando alg\'un denominador com\'un.
Creencias en realidades supernaturales y ocurrencias pueden no ser com\'un en todas las religiones, a\'un asi hay una clara divisi\'on entre diferentes aspectos de la vida, ciertos comportacmientos y cosas fi\'isicas.
En el pasado, argumentaba, la religi\'on habia sido el cimiento de la sociedad.
Sus definiciones de religi\'on, apoyadas por antrop\'ologos de hoy en d\'ia, era ``La religion es un sistema unificado de creencias y pr\'acticas relativas a cosas sagradas''.\\

Durkheim creia que la sociedad tenia que estar presente en el individuo.
Vi\'o en la religi\'on un mecanismo que fortalecia al orden social.
Pens\'o que la religi\'on hab\'ia sido un cimiento para la sociedad en el pasado, pero que el colapso de \'esta no llevar\'ia a una ``implosi\'on'' moral.
Durkheim estaba espec\'ificamente interesado en la religion como una experiencia de comunidad mas que una individual.
\'El incl\'uso dijo que el fen\'omeno religioso ocurria cuando se hacia una separaci\'on entre lo profano (la realidad de las actividades diarias) y lo sagrado (la realidad de los extraordinario y lo trascendental).
Un ejemplo de esto es la comuni\'on del vino, lo cual no es solo vino sino que representa la sangre de Jesucristo.
Adem\'as, en lugar de antepasados anteriores que intentaran reemplazar las religiones en decadencia, urg\'ian a las personas a unirse en una moralidad c\'ivica en las bases que ``somos lo que somos'' como resultado de la sociedad.\\

As\'i, Durkheim condenso las religiones en cuatro funciones mayores:
\begin{enumerate}
	\item Disciplinarias, forzando o administrando disciplina
	\item Cohesivas, juntando a las personas, generando fuertes lazos.
	\item Vitalizadoras, que aumentan el esp\'iritu, vitalizan y suben el \'animo.
	\item Euf\'oricas, con buenos sentimientos, felicidad, confidencia, bienestar.
\end{enumerate}
\newpage
