\vspace{1cm}
Por medio del presente informe se desarrollará en profundidad,
el legado del trabajo de \emph{Émile Durkheim}, analizando sus
ideas principales, rescatando así los factores negativos y positivos
de su trabajo.
\vspace{1cm}

Para tener una idea global de la importancia del personaje histórico
analizado a continuación, tenemos que comprender primeramente que es
uno de los fundadores de la sociología moderna, ya que a través de sus
esfuerzos en éste ámbito, se logró comprender con claridad las diferencias
entre la \emph{Sicología} y la \emph{Sociología}.
\vspace{1cm}

Present\'o variadas ideas basadas en el estudio de la sociedad,
entre ellas, vale destacar la del ``trabajo social'' y la idea de ``hechos sociales'', el cual mediante el método científico realizó un estudio a la sociedad, tratando de eliminar el sentimentalismo e ilusionismo, de dicho aspecto.
\vspace{1cm}

Sobre las bases de nuestro trabajo, cambiamos la estructura de los informes entregados anteriormente y los complementamos con mayor detalle,
 adem\'as de aportar con el contexto hist\'orico y filos\'ofico de la \'epoca para poder comprender mejor los antecedentes del trabajo realizado por Durkheim.
 Tambi\'en, agregamos las obras del autor junto con un peque\~no resumen de cada una de ellas.

\newpage
