\subsection{Contexto Hist\'orico}

\begin{enumerate}
\item \textbf{El\ segundo\ imperio\ de\ Napole\'on}

 Comenz\'o en el plebiscito que proclam\'o emperador a Luis Napole\'on Bonaparte (1852). Su gobierno se apoy\'o en el clero, en la alta burgues\'ia y en la poblaci\'on rural. Sigui\'o una pol\'itica empe\~nada en realzar el prestigio del pa\'is y del r\'egimen; en esa \'epoca, Haussmann embelleci\'o la capital. En pol\'itica exterior se inici\'o la conquista del Senegal, se penetr\'o en Argelia y se intervino en Indochina. Tambi\'en se apoy\'o a Italia en su lucha contra Austria por la unificaci\'on y se impuso a Maximiliano I como emperador de M\'exico (1862-1867). El r\'egimen se basaba en la prosperidad econ\'omica, pero \'esta entr\'o en crisis en 1870 cuando la desastrosa derrota de Sedan en la guerra francoprusiana (1870) puso fin al imperio, proclam\'andose la rep\'ublica.

\item \textbf{La\ tercera\ rep\'ublica}

La comuna revolucionaria de Par\'is, que hab\'ia surgido aprovechando el vac\'io del poder (1871), fue aplastada. Se elabor\'o la constituci\'on de 1875, de car\'acter conservador, como correspond\'ia al miedo de la burgues\'ia tras la revoluci\'on de 1871. Sin embargo, el movimiento obrero obtendr\'ia m\'as tarde la legalizaci\'on de los sindicatos (1884). En 1902 el bloque de izquierdas triunf\'o en las elecciones; se separ\'o la Iglesia del estado (1905), se difundi\'o la ense\~nanza laica y se intent\'o hacer frente a los problemas sociales mediante una legislaci\'on protectora de los trabajadores. En el exterior, durante el per\'iodo 1870-1914 se forj\'o un imperio colonial en \'Africa e Indochina, que fue el segundo en extensi\'on despu\'es del de Gran Breta\~na y mejor\'o la imagen del pa\'is.

La pugna con Alemania, que le hab\'ia arrebatado Alsacia y Lorena (1871), llev\'o a la Entente francobrit\'anica de 1904, al tratado franco-ruso de 1907 y a la Triple Entente (1907). Estas alianzas, tras diversas tensiones en las colonias y en la regi\'on balc\'anica, solucionadas al borde del conflicto armado, llevaron a la primera guerra mundial contra los imperios centrales: Alemania y Austria. Alemania ocup\'o B\'elgica e invadi\'o Francia buscando ocupar Par\'is, pero la invasi\'on rusa de Prusia y la victoria del Marne (1914) estabilizaron el frente y dieron paso a la carrera hacia el mar y a la guerra de trincheras. Las posiciones quedaron pr\'acticamente inamovibles hasta el final de la guerra, con encarnizadas y est\'eriles luchas por romper el frente; el uso de las ametralladoras y las alambradas hac\'ia inadecuados los asaltos a la bayoneta ordenados por los mandos franc\'es y alem\'an. En 1917 la intervenci\'on estadounidense empez\'o a llegar a suelo franc\'es y en 1918 Alemania fue derrotada.
\end{enumerate}
\newpage
