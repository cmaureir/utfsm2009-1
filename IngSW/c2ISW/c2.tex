\begin{enumerate}
	\item Explique brevemente 3 lecciones importantes aprendidas en este curso para su futuro desempeño profesional como ingeniero de software.\\
	\textbf{R:}
	\textcolor{red}{Ignacio}

	\item Considerando que la estimación de esfuerzo de software es una mezcla entre arte y ciencia, explique brevemente 3 acciones concretas que ejecutaría como jefe de proyecto para lograr que sea cada vez más ciencia y menos arte.\\
	\textbf{R:} En primer lugar, estableceria una tecnica para poder controlar el proyecto, sea:
                \begin{enumerate}
                        \item Carta Gantt
                        \item Division del Trabajo
                        \item Pert/CPM
                \end{enumerate}
		    Por otra parte organizaria la empresa de una forma, que puede ser:
                \begin{enumerate}
                        \item Centralizada
                        \item Descentralizada
                        \item Mixta
                \end{enumerate}
		    Por ultimo trabajaria de tal forma de controlar los riesgos que pueda tener el producto
                \begin{enumerate}
                        \item Identificacion: lista de riesgos especificos con probabilidad de comprometer el exito
                        \item Analisis: Evaluacion de magnitud y probabilidad de perdida
                        \item Priorizacion: Ranking de riesgos
			\item Planificacion de gestion: Preparacion y coordinacion de planes, para cada riesgo
			\item Mitigacion: eliminacion o solucion
			\item Monitoreo: Seguimiento del riesgo
                \end{enumerate}
	\textcolor{red}{Ignacio}

	\item Explique qué hace que la estimación del esfuerzo de desarrollo de software sea más un arte que una ciencia (por lo menos por ahora) y explique cómo se enfrenta esta situación.\\
	\textbf{R:} Esto se basa en 2 cosas:
                \begin{enumerate}
                        \item Factorez de incerteza:
				\begin{itemize}
					\item Complejidad
					\item Tamano
					\item Estructuracion
				\end{itemize}

                        \item Bases(Datos):
                                \begin{itemize}
                                        \item Datos Historicos
                                        \item Modelos
                                        \item Experiencia
					\item Sentido Comun
                                \end{itemize}
                \end{enumerate}
	\textcolor{red}{Ignacio}

	\item  Una de las mejores prácticas propuestas para la gestión de proyectos de software es “gestión consciente de las personas” que apunta a hacer un uso efectivo de los recursos de personal, explique brevemente a) en qué consiste la práctica, y b) cómo influye en la conformación de un equipo de trabajo.\\
	\textbf{R:} \textbf{A)} Consiste en organizar el personal al asignar roles y responsabilidades para lograr metas, para facilitar esto se busca una meta en comun, esto se hace principalmente para lograr tener una cohecion en el equipo, formar un equipo de trabajo. Diversos factores influyen al tomar estas deciciones:
                 \begin{itemize}
                         \item Tamano y duracion del proyecto
                         \item Naturaleza de las Tareas
                         \item Comunicacion(Cantidad de personas)
                         \item Relacion: Complejidad del problema - Tamano del equipo
                 \end{itemize}
		    \textbf{B)} Para gestionar bien el personal y formar un buen equipo de trabajo, se nececitan personas con caracteristicas como soportar stress, adaptabilidad, ingenio, creatividad. De esta forma el querer organizar y usar efectivamente los recursos del personal hay que hacer una seleccion previa para captar a los mejores y que cumplan los requerimientos puestos por la empresa como el trabajo o proyecto en el que trabajara. Algunas diferencias entre las personas:
                 \begin{itemize}
                         \item Capacidad de realizar su trabajo
                         \item Interes en su trabajo
                         \item Experiencia: Aplicaciones, herraminetas, tecnicas y desarrollo
                         \item Capacidad de : Entrenamiento, comunicacion y compartir responsabilidades
                 \end{itemize}
	\textcolor{red}{Ignacio}

	\item Explique brevemente:
		\begin{enumerate}
			\item las distintas visiones de calidad de software
			\item verificación de software
			\item validación de software
			\item ¿qué visión de calidad sólo se focaliza en validación y no en verificación?
		\end{enumerate}
	\textbf{R:}
                \begin{enumerate}
                        \item las distintas visiones de calidad de software
                 \begin{itemize}
                         \item Trascendental: Reconocida pero no definida
                         \item Usuario: Grado de adecuacion al proposito
                         \item Productor: Conformidad con la especificacion
                         \item Basada en valor: Cuanto el cliente esta dispuesto a pagar
                 \end{itemize}

                        \item verificación de softwaree: Se refiere a si el producto se esta contruyendo \textbf{de la forma correcta.}
                        \item validación de softwaree: Se refiere a si se esta contruyendo \textbf{El producto Correcto}
                        \item La vision que se focaliza en validacion y no en verificacion es la de \textbf{Usuario} ya que hacia el va dirigido el producto, y es el, el unico que es lo que nececita y por lo tanto saber si el producto es el correcto para satisfacer sus nesecidades.
                \end{enumerate}

	\textcolor{red}{Ignacio}

	\item Explique en forma breve y precisa
		\begin{enumerate}
			\item cuál es el objetivo específico del testing de software\\
	\textbf{R:}
El objetivo del testing de software es encontrar defectos, se debe mostrar que algo es incorrecto.
			\item cuál es la actividad clave para el éxito del testing\\
	\textbf{R:}
\textbf{...}
Supongo que el revisar el software para encontrar defectos en el.
			\item ilústrelo con un ejemplo sencillo.\\
	\textbf{R:}
\textbf{probamos y encontramos que algo no hace/hace cosas que debería/no debería hacer, registramos la
falla, para luego buscar el error que la provoca y arreglarlo. }
		\end{enumerate}
	\textbf{R:}
	\textcolor{red}{Rodrigo}

	\item En relación al método ágil SCRUM,
		\begin{enumerate}
			\item comente la siguiente aseveración “SCRUM no es realmente una metodología de desarrollo de software, sino de gestión de proyectos”\\
	\textbf{R:}
				Scrum es un framework simple que puede ser implementado en unos pocos dias, con un enfoque para
				manejar problemas complejos, un ambiente para soportar auto-organización, creatividad y emergencia. Es un
				esfuerzo colaborativo que involucra desarrolladores y clientes en un dialogo constante.
				Sus puntos claves son el proceso empírico (ciclos ``just-in-time'' de revisión y adaptación) y la
				auto-organización (el equipo se autogestiona y organiza alrededor de metas).
				En fin, se puede concluir que si se puede aplicar a una amplia gamma de proyectos, ya
				que su aplicación no está limitada únicamente al desarrollo de software.

			\item ¿cuál es el rol del SCRUM Master?(es lo mismo que ¿cuál es el rol del dueño del producto o product owner?)\\
	\textbf{R:}
				El product Owner gestiona la visión y los requerimientos del proyecto. Es el responsable de construir y
				difundir la visión del producto, mejorar el ROI (Return On Investment), conciliar las expectativas del
				cliente y de los stakeholders, trazar el rumbo a seguir y planificar las entregas, coordinar y priorizar
				el Product Backlog, proveer requerimientos claros y testeables al equipo, colaborar con el cliente y con
				el equipo para asegurar que se cumplan las metas y para aceptar el software al final de cada iteración.
				
				El scrum Master (facilitador) es el que gestiona el proceso, es líder servidor del equipo, responsable de
				guiar, crear un ambiente de confianza, facilitar las reuniones, hacer las preguntas difíciles, remover
				impedimentos, hacer visibles los problemas, mantener al proceso avanzando, sociabilizar Scrum y Agile con
				el resto de la organización, educar a Clientes y otros stakeholders.

				\textbf{.. probablemente ahora pregunten sobre ``The Team'':}\\
				El equipo se gestiona a sí mismo, se maneja desde una perspectiva táctica. Es responsable por estimar el
				tamaño de los ítems del Product Backlog, tomar decisiones de diseño e implementación, y comprometerse a
				entregar incrementos de software. Sigue su propio progreso (con la ayuda del Scrum Master), es autónomo y
				auto-organizado, pero debe responder ante el Product Owner para entregar lo prometido.

			\item ¿cuál es el objetivo de la reunión diaria o daily scrum?\\
	\textbf{R:}
				El objetivo de esta reunión es ponerse al día del trabajo realizado por el equipo, el progreso,
				intenciones e impedimentos, ayudando a controlarel avance y a actualizar el Task Board.
				Esta debe realizarse todos los días y no debe durar más allá de 15 minutos.\\
				\textbf{Estudiar todos los tipos de Reuniones y artefactos de Scrum}
		\end{enumerate}
	\textcolor{red}{Rodrigo}

	\item  ¿Qué significa que un proceso de software sea gerenciable? ¿Por qué eso es bueno? ¿Qué relación con esto tiene el modelo CMMI?\\
	\textbf{R:}
\textbf{NO LO ENCONTRE, ¿Se referirá a los niveles de madurez de los equipos de desarrollo?} Nivel 2: Managed (gestionable,
gerenciable), requiere una etapa de gestión de requerimientos, plan, control y monitoreo del proyecto,
gestion de aceptación de proveedores, mediciones y análisis, aseguramiento de la calidad de los procesos
y del producto, y finalmente un configuration management.
	\textcolor{red}{Rodrigo}

	\item Considere la herramienta WBS (Work Breakdown Structure) y explique en forma breve y precisa (en el contexto de gestión de proyectos de software):
		\begin{enumerate}
			\item ¿qué es?
			\item ¿para qué sirve?
			\item ¿qué práctica(s) específica(s) apoya?
			\item ¿por qué le parece que es importante utilizarla cuando se desarrolla software?
		\end{enumerate}
	\textbf{R:}
		WBS = Estructura desglosada de trabajo. Es una forma de controlar proyectos, asi como cata gantt y PERT/CPM.Permite
		definir el trabajo de lo general a lo particular en la etapa de planeación y cuantificar avances y
		recursos de lo particular a lo general, en la etapa de seguimiento y control de proyecto.
	\textcolor{red}{Rodrigo}

	\item  ¿Cuáles son los planes útiles para gestionar el riesgo? Ilustre con un ejemplo cada uno.\\
	\textbf{R:}
Está la Evaluación de riesgos, que contempla:
\begin{enumerate}
	\item Identificacion: El  programa esta enfocado  a un rango de edad equivocado
	\item Análisiss El rango de edad de la aplicacion, gracias a su vocabulario,esta por encima de lo acordado
	\item Priorizacion: Primero es cambiar el vocabulario utilizado, segundo modificar la interfaz de tal forma de ser mas acorde.
\end{enumerate}

y está el Control de riesgos, que contempla:
\begin{enumerate}
	\item Planificación de Gestión (Mitigación y Contingenciacia): Bucaremos a profesores que nos puedan ayudar amej mejorar el vocabulario utilizado
	\item Resolución/Mitigaciion: Programadores utilizaran el material dado para cambiar enp primera instancia el vocabulario y por otra parte la interfaz
	\item Monitoreo: Ver como afecta alos usuario el cambio.
\end{enumerate}

La idea es describir los posibles riesgos de recursos, técnicos, o del negocio implicados en
el proyecto, y formular un plan para abordar los posibles riesgos, con medidas de mitigación y correctivas
para afrontar cada uno de ellos. Sirve de punto principal para la programar las actividades que deben
realizarse y con base en este documento se deben plantear las iteraciones a ser realizadas.

Un Plan de Gestión del Riesgo debe ser documentado a comienzos del proyecto, durante la fase de inicio.
El plan es emprendido ante la fase de elaboración para asegurar que ninguno los riesgos identificados
sean direccionados durante la misma fase de elaboración. Apenas el plan haya sido documentado, el proceso
de prevención de riesgos estará ocupado para monitorear y controlar la probabilidad y el impacto de los
riesgos sobre el proyecto. 
\textbf{No tengo idea a que ejemplos se refiere}
	\textcolor{red}{Rodrigo}

	\item Su empresa ha sido contratada para desarrollar software que será especificado por analistas en Tanzania (cuyo idioma materno obviamente es swahili). Explique qué prácticas usaría para superar los riesgos del caso (si no hay riesgos, dígalo).\\
	\textbf{Si es que utilizaran el idioma de ese pais para la especificacion del software a desarrollar, claramente tenemos un problema, ya que es un riesgo potencial que se va a llevar a cabo si o si. Debemos: (ver diagraa de gestion de proyectos, camino de un problema}
	\begin{itemize}
	\item Evaluar el problema
	\item Identificar el tama\~no del problema
	\item Estimar los esfuerzos a utilizar para solucionar el problema
	\item Luego hacer una division del trabajo mas esfuerzo, teniendo en cuenta la contingencia, costos, escala de tiempo y perfil del equipo
	\item Generamos un plan
	\end{itemize}
	\textcolor{red}{Gabriel}

	\item Explique la relación entre un plan de contingencia de riesgos y un gatillador (trigger). Ilustre con un ejemplo\\

	\textbf{Un Plan de contingencias es un instrumento de gestion para el buen gobierno de las TICs en el dominio del soporte y el desempe\~no. Dicho plan contiene las medidas tecnicas, humanas y organizativas necesarias para garantizar la continuidad del negocio y las operaciones de una compa\~n\'ia. El plan nace de un analisis de riesgo que pudiesen afectar la continuidad del negocio, el trigger o gatillador es la descripcion de la situacion de riesgo, en la que se podria encontrar el negocio, la que provocaria que se lleve a cabo el plan de contingencia.}

	\textbf{Plan: Subcontratar desarrolladores de GUI para LMN Corp. y aceptar el aumento de nuestros costos, 25.000. LMN tiene un contrato de nivel de esfuerzo ABC y pueden apoyar con 1 semana de aviso. Gatillador: Si los expertos internos no esten involucrados y el entrenamiento no ha concluido a la fecha 7/30/95}
	\textcolor{red}{Gabriel}

	\item Defina y compare verificación y validación. ¿Qué hay que hacer primero?\\
	\textbf{Verificacion consiste en identificar si se esta construyendo el producto correctamente. Validacion consiste en identificar si lo que se construye, es correcto. La diferencia radica en que el primero solo se preocupa del desarrollo del producto en si, en cambio la validacion se cuestiona si es correcto su realizacion como producto.}
	\textbf{Supongo debemos saber si construimos algo correcto, que ver si lo construimos correctamente, digo yo :P}
	\textcolor{red}{Gabriel}

	\item  El jefe de un proyecto estima que quedan unos 200 defectos en un programa a ser entregado, y el equipo de prueba (testing) encuentra 185, de modo que el equipo prepara la entrega final. ¿Qué consejo le daría Ud al jefe de proyecto?\\
	\textbf{Le diria que esta hablando puras pescas, por que los defectos que detecto el equipo de prueba, aplicando testing y encontrandose con fallas, son los que importan, no hay defecto si el sistema no puede fallar.}

	\textcolor{red}{Gabriel}

	\item Los modelos de calidad incluyen factores, criterios y métricas. Describa estos 3 elementos para la noción de usabilidad.\\
	\begin{itemize}
	\item Factores: Incluyen 3 categorias. Revision del producto: los cualoes incluyen factores como facilidad de mantenimiento, flexibilidad y facilidad de prueba. Transicion del producto: Portabilidad, Reusabilidad, Interoperabilidad. Operacion del producto: correccion (hace lo que quiero), confiabilidad, eficiencia, integridad (es seguro), facilidad de uso.
	\item Criterios: No encontre eso, quizas esta implicito en algo, si lo encuentro lo pongo.
	\item Metricas: Consiste en determinar que medir y para que medirlo, como tama\~no de datos, algoritmos, funcion, human-oriented: Metircas de tama\~no, Metricas funcionales, Metricas de productividad, Metricas de calidad.
	\end{itemize}
	\textcolor{red}{Gabriel}

	\item  En el contexto de Gestión de Configuración de Software (SCM) explique:
		\begin{enumerate}
			\item ¿qué es un baseline?\\
			\textbf{R:}
			Es un producto formalmente revisado y aprovado, base para futuros desarrollos.
			\item ¿cómo se modifica un baseline durante el ciclo de vida del producto de software?\\
			\textbf{R:}
			Puede ser modificado solo a través de un procedimiento formal de control de cambios.
			\item ¿cuál es la diferencia clave entre SCM y otras fases del desarrollo de software?\\
			\textbf{R:}
			Consiste en identificar, organizar y controlar los cambios al producto que se desarrolla y cubre el proceso completo.
		\end{enumerate}
	\item  ¿Es realmente posible combinar agilismo y el Rational Unified Process (RUP)? Si no lo es, explique porqué, y si lo es, dé un ejemplo.\\
	\textbf{R:}
	RUP puede usarse en un estilo muy tradicional de cascada o de una manera agil. Como resultado usted puede usar el RUP como un proceso ágil, o como un proceso pesado - todo depende de cómo lo adapte a su ambiente. 	


	\item  La empresa A tiene un nivel CMMI 5, mientras la empresa B tiene un nivel CMMI 3. Usted debe contratar a una de ellas.
		\begin{enumerate}
			\item ¿qué le dice la información del nivel CMMI?\\
			\textbf{R:}
			CCMI significa Capability Maturity Model Integration.
			Es un modelo para la mejora y evaluación de procesos para el desarrollo, mantenimiento y operación de sistemas de software.
			
			Un nivel CMMI 5 me dice lo siguiente:\\
			\emph{Optimizando: Además de ser un proceso cuantitativamente gestionado, de forma sistemática se revisa y modifica o cambia para adaptarlo a los objetivos del negocio. Mejora continua}\\
			Un nivel CMMI 3 analogamente:\\
			\emph{Definido: Además de ser un proceso gestionado se ajusta a la política de procesos que existe en la organización, alineada con las directivas de la empresa.}
				
			\item ¿qué diferencia presenta la última versión (v1.2) de CMMI al respecto?\\
			\textbf{R:}
			CCMI v1.2: 2006 en adelante, además del desarrollo (CMMI-DEV) incorpora adquisición (CMMI-ACQ) y servicios (CMMI-SVC).\\
			CMMI-DEV, en él se tratan procesos de desarrollo de productos y servicios.\\
			CMMI-ACQ, en él se tratan la gestión de la cadena de suministro, adquisición y contratación externa en los procesos del gobierno y la industria.\\
			CMMI-SVC, actualmente un borrador, está diseñado para cubrir todas las actividades que requieren gestionar, establecer y entregar Servicios.\\
			
		\end{enumerate}
	\item Una empresa lo invita a Ud a auditar un proyecto en problemas, ya que una prueba final del sistema en desarrollo encontró sólo 5 defectos, y esto le parece muy extraño al gerente. ¿Qué le podría decir Ud. al gerente aún antes de auditar las pruebas?\\
	\textbf{R:}
	\textcolor{red}{Cristián}

	\item ¿Qué es el retrabajo, y qué tiene que ver con la calidad?\\
	\textbf{R:}
	Se considera retrabajo al trabajo que se hace a causa de no haber realizado el “trabajo”  correctamente la primera vez.\\
	También se considera retrabajo los cambios continuos que se hacen y el trabajo duplicado entre personas.\\
	La causa más frecuente de retrabajo es la necesidad de hacer correcciones para resolver defectos o no cumplimientos de los estándares establecidos.\\
	Las  métricas de retrabajo puede ayudar al equipo de SQA a demostrar la conveniencia de mejores planes de proyectos teniendo más cuidado y prestando mayor atención a los requerimientos.\\
	Definimos tambien el Costo de calidad. (COQ)\\
	COQ =  COA + COF\\
	COA: Cost of Quality, costo de recursos asignados para prevenir errores y alcanzar la calidad.\\
	COF: Cost of Failure, Costo de recursos utilizados por que la calidad no fue alcanzada (este es el retrabajo)\\

	\item Mencione los tipos de mantención de software, y dé un ejemplo de cada tipo para la mantención del SIGA.\\
	\textbf{R:}
	\textcolor{red}{El que termine primero}

	\item Su empresa ha decidido subcontratar el desarrollar de un producto a una empresa en Thailandia (cuyo idioma materno obviamente es thai). ¿Qué recomendaciones específicas le haría Ud. al jefe de proyecto?\\
	\textbf{R:}
	\textcolor{red}{El que termine primero}

\end{enumerate}
