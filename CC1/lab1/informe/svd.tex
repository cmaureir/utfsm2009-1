\begin{enumerate}
	\item Implemente un algoritmo que calcule la factorización de svd.

	\textbf{R:}\\
	\begin{verbatim}

	function [U,S,V] = SVD(A)
		H = A'*A;
		[V,D] = eig(H);
		for i = 1:size(V)(1)
			D2(i,i) = sqrt(abs(D(i,i)));
		end
		for i = 1:size(D2)(1)
			j = size(D2)(1)+1-i;
			S(j,j) = D2(i,i);
		end
		for j = 1:size(V)(1)
			norma = 0;
			for i = 1:size(V)(1)
				norma = V(i,j)*V(i,j) + norma;
			end
			norma = sqrt(norma);
			for i = 1:size(V)(1)
				V(i,j) = V(i,j)/norma;
			end
		end
		s = size(V)(2);
		for j = 1:s
			for i = 1:s
				tmp(i,j) = V(i, s+1-j);
			end
		end
		V = tmp;
		U = A*V*inv(S);
	endfunction
	\end{verbatim}
	\newpage
	\item Compare su implementación con la que provee software como Octave. Para esto genere matrices
de tamaño superior a $101x101$ de forma aleatoria.

	\textbf{R:}\\

	Claramente la implementación que provee el software Octave, es mucho mas eficiente que el nuestro,
	especialmente en el tiempo que toma el calculo de la factorización. Además, notamos algunas
	diferencias (algunos cambios de signo), debido a la implementación de la función eig() que
	provee Octave, salvo en las matrices $\sum$, que contienen valores singulares idénticos.

\end{enumerate}

