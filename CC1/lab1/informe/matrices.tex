\begin{enumerate}
	\item Genere una matriz de $7x7$ que contenga sólo números
	primos ordenados de menor a mayor.\\
	\textbf{R:}\\
\begin{center}
$
\begin{bmatrix}
     2 &    3 &    5 &    7 &   11 &   13 &   17\\ 
    19 &   23 &   29 &   31 &   37 &   41 &   43\\
    47 &   53 &   59 &   61 &   67 &   71 &   73\\
    79 &   83 &   89 &   97 &  101 &  103 &  107\\
   109 &  113 &  127 &  131 &  137 &  139 &  149\\
   151 &  157 &  163 &  167 &  173 &  179 &  181\\
   191 &  193 &  197 &  199 &  211 &  223 &  227
\end{bmatrix}
$
\end{center}
	\item Encuentre los valores propios de la matriz generada,
	 usando la función provista por el software.\\
	\textbf{R:}\\
$
\lambda_1 = 741.95179 + 0.00000i\\ 
\lambda_2 = -26.71104 + 0.00000i\\
\lambda_3 =   5.00551 + 1.16550i\\
\lambda_4 =   5.00551 - 1.16550i\\
\lambda_5 =   0.78161 + 0.00000i\\
\lambda_6 =  -1.01669 + 0.04980i\\
\lambda_7 =  -1.01669 - 0.04980i\\
$
	\item Implemente un algoritmo que encuentre los valores propios
	de la matriz generada.\\
	\textbf{R:}\\
	El algoritmo que utilizamos,
	está basado en obtener el polinomio característico,
	y luego obtener las raíces de dicha ecuación.\\
	\begin{verbatim}
	octave:1> roots(poly(m))
	\end{verbatim}
	Donde $m$, es la matriz con la cual estamos trabajando.\\
	$poly(m)$ obtiene un vector con los coeficientes del polinomio característico y
	$roots(m)$ obtiene las raíces de un polinomio determinado.
	
	\item Compare los resultados obtenidos de su función contra la
	función provista por el software.\\
	\textbf{R:}\\
	Los resultados para nuestra matriz de $7x7$
	en ambos casos fueron iguales,
	ésto es debido a que de las dos formas se utilizan funciones provistas por octave.\\


	\item Poniendo a prueba a los algoritmos: compare los tiempos de respuestas de su algoritmo,
		realice varios experimentos aumentando el tamaño de la matriz progresivamente hasta un
		tamaño razonable.
	\begin{enumerate}
 		\item ¿Cómo se comporta su algoritmo a medida que el tamaño de la matriz aumenta?\\
		\textbf{R:}\\
		Nuestro algoritmo soporto sólo hasta una matriz de $507x507$,
		ésto se debe a las limitaciones que tiene el comando $roots()$,
		lo cuál nos llamo mucho la atención ya que como los provee el mismo programa,
		uno piensa que son 100\% eficientes, pero si nos damos cuenta
		resolver una ecuación de grado $507$ es todo un reto.
		\begin{verbatim}
		roots(poly(rand(507)))
		\end{verbatim}
		Para valores aceptados muy grandes,
		se demora aproximadamente $5$ segundos en poder entregar la solución.

		\item ¿Sucede lo mismo con el algoritmo que utiliza la función del software, por qué?\\
		\textbf{R:}\\
		El algoritmo del software $eig()$ se demora mucho menos en poder obtener los
		valores propios, alrededor de $2$ segundos.
		Lo cual claramente es un signo de que cuando un algoritmo está enfocado a una tarea
		en especial, es mucho mas eficiente comparado con la forma $inadecuada$ que pudimos
		plantear para obtener el resultado.

		\item ¿Qué es un tamaño razonable?
		\textbf{R:}\\
		Un tamaño razonable con respecto a éste problema,
		es un tamaño que transforma la simple tarea de realizar
		\emph{a mano} algún algoritmo para obtener valores propios,
		en un ejercicio \emph{inhumano} y casi imposible de realizar.

		Nadie en su sano juicio,
		sería capaz de realizar el algoritmo con el polinomio característico
		para obtener valores propios de una matriz de $500x500$

	\end{enumerate}
	\item Encuentre los vectores linealmente independientes de la matriz generada en $(2.1)$.\\
	\textbf{R:}\\
	Como la matriz que formamos está compuesta sólo con números primos,
	no existe la posibilidad de que existan múltiplos de algunos vectores,
	y que lleguemos a la conclusión de que dos vectores son \emph{linealmente dependiente}.

	Por lo tanto todos los vectores de la matriz son \emph{linealmente independientes}.

	Una manera de comprobarlo,
	es operando la matriz hasta poder llegar a su forma irreductible.
	Ésto es posible mediante la siguiente sentencia de \emph{octave}\\
	\begin{verbatim}
	octave:1> ((rref(m')))'
	\end{verbatim}
	Lo que quiere decir, que trasponemos la matriz, realizamos operaciones filas,
	finalmente la trasponemos nuevamente, para dejarla en su estado inicial, y el
	resultado que nos estrega es la matriz de identidad de dimensiones $7x7$,
	lo que quiere decir, que llegamos a las bases canónicas, comprobando así
	que \emph{todos} sus vectores son \emph{linealmente independientes}
	
	
	

	\item ¿Puede dar un ejemplo de una matriz cuyos vectores sean ortogonales pero que no sean
		linealmente independientes? Demuestre.\\
		\textbf{R:}\\
		Supongamos que existe un conjunto ortogonal $ x_1,x_2,\cdots,x_k$ y construyamos una combinación
		lineal nula de ellos:

		$\displaystyle \alpha_1x_1+\alpha_2x_2+\cdots+\alpha_nx_n=0 $

		Si seleccionamos un vector cualquiera $ x_j$ y efectuamos el producto interno a cada lado
		de la ecuación tenemos

		$\displaystyle \alpha_1P(x_1,x_j)+\alpha_2P(x_2,x_j)+\cdots+\alpha_nP(x_n,x_j)=0 $

		Como se trata de un sistema ortogonal, el único $ P(x_i,x_j)\neq 0$ es $ P(x_j,x_j)=P_i$, y
		por lo tanto

		$\displaystyle \alpha_jP(x_j,x_j)=\alpha_jP_i=0 $

		Y por tanto

		$\displaystyle \alpha_j=0 $

		Por lo que podemos concluir que la única combinación lineal nula de los vectores es la
		que tiene coeficientes nulos, es decir, los vectores son linealmente independientes.

\end{enumerate}


