\scriptsize
\subsection{C\'odigo base}
\begin{verbatimtab}
#include <math.h>
#include <stdlib.h>
#include <stdio.h>
#include "global.h"

int x[N], y[N];
float theta[N];
int count[N];
float g[(int)(N*1.5)];

int main(int argc, char **argv)
{
	for(i=0;i<N;i++)
	{
		x[i] = i;
		y[i] = i;
		theta[i] = 6*random();
	}

	//	Almacenamos tiempo capturado
	initialTime = getTime();

	//	Now, accumulate data for all pairs of
	//	points (i,j).
	for(i=0; i<N; ++i)	//for each i < N
		for(j=i+1; j<N; ++j)	//for each j < N
		{
			Dx = x[i]-x[j];
			Dy = y[i]-y[j];
			r = sqrt(Dx*Dx + Dy*Dy);
			g[r] += cos(6*(theta[i]-theta[j]));
			++count[r];
		}

	for (r = 0; r < MAX_R; r++)
		g[r] = g[r] / count[r];

	//	Calculando lo que ha tomado el calculo
	finalTime = getTime();

	//	Imprimimos resultados
	results();
	return 0;

}
\end{verbatimtab}


\subsection{C\'odigo base mejorado mejorada}
\begin{verbatimtab}
#include <math.h>
#include <stdlib.h>
#include <stdio.h>
#include "global.h"

int x[N], y[N], ii;
float theta[N];
int count[N];
float g[(int)(N*1.5)];
float sin6[N], cos6[N];

int main(int argc, char **argv)
{
	for(i=0;i<N;i++)
	{
		x[i] = i;
		y[i] = i;
		theta[i] = 6*random();
	}

	//	Almacenamos tiempo capturado
	initialTime = getTime();

	for(ii=0; ii<N; ++ii)
	{ 	
		sin6[ii] = sin(6*theta[ii]);
		cos6[ii] = cos(6*theta[ii]);
	}
	
	for(i=0; i<N; ++i)
		for(j=i+1; j<N; ++j)
		{
		Dx = x[i]-x[j];
		Dy = y[i]-y[j];
		r = sqrt(Dx*Dx+Dy*Dy);
		g[r] += cos6[i]*cos6[j]+sin6[i]*sin6[j];
		++count[r];
		}
	
	for (r = 0; r < MAX_R; r++)
		g[r] = g[r] / count[r];

	//	Calculando lo que ha tomado el calculo
	finalTime = getTime();

	//	Imprimimos resultados
	results();
	return 0;

}
\end{verbatimtab}

\subsection{C\'odigo con localidad espacial mejorada}
\begin{verbatimtab}
#include <math.h>
#include <stdio.h>
#include <stdlib.h>
#include <time.h>
#include "global.h"

struct DataStruct
{
	int x, y;
	float cos6, sin6;
} data[N];

struct AccumulationStruct
{
	int count;
	float g;
} accum[(int)(N*1.5)];

int ii;
float theta[N];

int main(int argc, char **argv)
{
	for(i=0;i<N;i++)
	{
		data[i].x = i;
		data[i].y = i;
		theta[i] = 6*random();
	}


	//	Almacenamos tiempo capturado
	initialTime = getTime();

	for (ii = 0; ii < N; ii++)
	{
		data[ii].cos6 = cos(6*theta[ii]);
		data[ii].sin6 = sin(6*theta[ii]);
	}
	
	for(i=0; i<N; ++i)	
		for(j=i+1; j<N; ++j)
		{
			Dx = data[i].x - data[j].x;
			Dy = data[i].y - data[j].y;
			r = sqrt(Dx*Dx + Dy*Dy);
			accum[r].g += data[i].cos6 * data[j].cos6 + data[i].sin6 * data[j].sin6;
			++accum[r].count;
		}

	for (r = 0; r < MAX_R; r++)
		accum[r].g = accum[r].g / accum[r].count;

	//	Calculando lo que ha tomado el calculo
	finalTime = getTime();
	
	//	Imprimimos resultados
	results();
	return 0;
}
\end{verbatimtab}

\subsection{C\'odigo con localidad espacial y temporal mejorada}
\begin{verbatimtab}
#include <math.h>
#include <stdio.h>
#include <stdlib.h>
#include <time.h>
#include "global.h"
#include "tabla.h"

struct DataStruct
{
	int x, y;
	float cos6, sin6;
} data[N];

struct AccumulationStruct
{
	int count;
	float g;
} accum[(int)(N*1.5)];

int ii;
int A,B,C,blocksize =1000;
int flag;
float theta[N];

int main(int argc, char **argv)
{
	for(i=0;i<N;i++)
	{
		data[i].x = i;
		data[i].y = i;
		theta[i] = 6*random();
	}

	//	Almacenamos tiempo capturado
	initialTime = getTime();

	for(ii = 0; ii < N; ii++)
	{
		data[ii].cos6 = cos(6*theta[ii]);
		data[ii].sin6 = sin(6*theta[ii]);
	}

	for(A=0; A<N; A+=blocksize)
	{
		C=A;
		for(B=A,flag=1; B<N; B+=blocksize,flag=0)
		{
			for(i=A; i<A+blocksize; ++i)
			{	
				if(flag)
				{
					C++;
					for(j=C; j<B+blocksize; ++j)
					{
						Dx = data[i].x - data[j].x;
						Dy = data[i].y - data[j].y;
						r = sqrt(Dx*Dx + Dy*Dy);
						accum[r].g += data[i].cos6 * data[j].cos6 + data[i].sin6 * data[j].sin6;
						++accum[r].count;
					}
				}
				else
					for(j=B; j<B+blocksize; ++j)
					{
						Dx = data[i].x - data[j].x;
						Dy = data[i].y - data[j].y;
						r = sqrt(Dx*Dx + Dy*Dy);
						accum[r].g += data[i].cos6 * data[j].cos6 + data[i].sin6 * data[j].sin6;
						++accum[r].count;
					}
			}
			C=B;
		}
	}

	for (r = 0; r < MAX_R; r++)
		accum[r].g = accum[r].g / accum[r].count;

	//	Calculando lo que ha tomado el calculo
	finalTime = getTime();
	
	//	Imprimimos resultados
	results();
	return 0;

}
\end{verbatimtab}
