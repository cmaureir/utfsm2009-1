\begin{description}
	\item[Memoria Principal:]
Está formada por bloques de circuitos integrados o chips capaces de almacenar, retener o ``memorizar''
información digital, es decir, valores binarios; a dichos bloques tiene acceso el microprocesador de la
computadora. El microprocesador la accede mediante el bus de direcciones. La MP es el núcleo del
sub-sistema de memoria de un computador, y posee una menor capacidad de almacenamiento que la memoria
secundaria, pero una velocidad muy superior.

	\item[Bus:]
Es un sistema digital que transfiere datos entre los componentes de un computador o
entre computadores. Están formado por cables o pistas en un circuito impreso, dispositivos como
resistencias y condensadores además de circuitos integrados. La función del Bus es la de permitir la
conexión lógica entre distintos subsistemas de un sistema digital, enviando datos entre dispositivos de
distintos ordenes: desde dentro de los mismos circuitos integrados, hasta equipos digitales completos que
forman parte de supercomputadoras.

	\item[Memoria Caché:]
Es un sistema especial de almacenamiento de alta velocidad. Puede ser tanto un área reservada de la
memoria principal como un dispositivo de almacenamiento de alta velocidad independiente. Hay dos tipos de
cache frecuentemente usados en las computadoras personales: memoria cache y cache de disco. Una memoria
cache, llamada también a veces almacenamiento cache o RAM cache, es una parte de memoria RAM estática de
alta velocidad (SRAM) más que la lenta y barata RAM dinámica (DRAM) usada como memoria principal. La
memoria cache es efectiva dado que los programas acceden una y otra vez a los mismos datos o
instrucciones. Guardando esta información en SRAM, la computadora evita acceder a la lenta DRAM.

	\item[Principio de Localidad:]
Los programas acceden una porción relativamente pequeña del
espacio de direcciones en un determinado instante de tiempo.

	\item[Localidad Espacial:]
Se refiere al espacio en donde son guardados los datos en la memoria. Es una propiedad en la que se basa
una de las políticas de extracción utilizadas para determinar cuándo y qué bloque de memoria principal
hay que traer a memoria cache. Si un item es referenciado, los items tenderán a ser referenciados a la
brevedad.

	\item[Localidad Temporal:]
Se refiere a la frecuencia con la cual son consultados ciertos bloques guardados en la memoria.
Si un item es referenciado, tenderá a ser referenciado nuevamente a la brevedad.

	\item[Latencia:]
En redes informáticas de datos se denomina latencia a la suma de retardos temporales dentro de una red.
Un retardo es producido por la demora en la propagación y transmisión de paquetes dentro de la red.

	\item[Ciclos de Reloj:]
La frecuencia de reloj indica la velocidad a la que un ordenador realiza sus operaciones más básicas,
como sumar dos números o transferir el valor de un registro a otro. Se mide en ciclos por segundo
(hercios).
	\item[CPU:]
Es el componente en una computadora digital que interpreta las instrucciones y procesa los datos
contenidos en los programas de la computadora. Las CPU proporcionan la característica fundamental de la
computadora digital (la programabilidad) y son uno de los componentes necesarios encontrados en las
computadoras de cualquier tiempo, junto con el almacenamiento primario y los dispositivos de
entrada/salida. 
\end{description}
